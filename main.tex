\documentclass[a4paper,12pt, openany]{book}  % Book class for thesis 
\usepackage[utf8]{inputenc}
\usepackage{graphicx}
\usepackage{geometry}
\geometry{left=2cm, right=2cm, top=2cm, bottom=2cm} 
\usepackage{setspace}
\setstretch{1}
\raggedbottom
\usepackage{titlesec}
\usepackage[toc,page]{appendix}
\titleformat{\subsection}[hang]{\normalfont\large\bfseries}{\thesubsection}{1em}{}
\titlespacing*{\subsection}{0pt}{\baselineskip}{0.5\baselineskip}
\usepackage[bottom]{footmisc}
\usepackage{ragged2e}
\usepackage{lmodern}
\usepackage{fancyhdr}
\usepackage{tocloft}
\usepackage{xcolor}
\usepackage[
    colorlinks=true,        % disables boxes around links
    urlcolor=teal,          % URL link color
    linkcolor=black,        % color of internal links (e.g., TOC)
    citecolor=blue
]{hyperref}
\usepackage[nameinlink,noabbrev]{cleveref}
\usepackage{array}
\usepackage[nottoc]{tocbibind}
\usepackage{booktabs}
\usepackage{float}
\usepackage[T1]{fontenc}
\usepackage{soul}  
\usepackage[font=bf,labelfont=bf,justification=raggedright,singlelinecheck=false]{caption}
%\captionsetup{font={bf,small}}
\captionsetup{font={bf,small}, labelfont=bf, justification=raggedright, singlelinecheck=false}
\usepackage{chngcntr}
\counterwithout{figure}{chapter}
\counterwithout{table}{chapter}

\usepackage{multirow, tabularx, xurl}

\usepackage[authordate,backend=biber,natbib]{biblatex-chicago}
\ExecuteBibliographyOptions{maxcitenames=2, mincitenames=1}
\DeclareDelimFormat{nameyeardelim}{\addcomma\space}
\addbibresource{references.bib}
\AtEveryCitekey{%
  \ifciteseen
    {}
    {\color{blue}}%
}
\renewbibmacro*{doi+eprint+url}{
  \printfield{doi}
}
\DeclareCiteCommand{\citep}
  {\begingroup(}  % Start a group, manually insert black opening parenthesis
  {\color{blue}\usebibmacro{citeindex}%
   \usebibmacro{cite}}
  {\multicitedelim}
  {\usebibmacro{postnote}\endgroup)}  % Close blue group, then black parenthesis
\DeclareFieldFormat{doi}{%
  DOI: \ifhyperref
    {\href{https://doi.org/#1}{#1}}
    {#1}}

% Page numbering setup (BOTTOM CENTER)
\pagestyle{fancy}
\fancyhf{} % Clear all headers and footers
\renewcommand{\headrulewidth}{0pt} % Remove header line
\fancyfoot[C]{\thepage} % Center page number in footer

% Apply same style to chapter start pages
\fancypagestyle{plain}{%
  \fancyhf{}
  \renewcommand{\headrulewidth}{0pt}
  \fancyfoot[C]{\thepage}
}

% Chapter formatting
\titleformat{\chapter}[display]
  {\normalfont\huge\bfseries}
  {\chaptertitlename\ \thechapter}{20pt}{\Huge} 


% Customize Table of Contents
\setlength{\cftbeforetoctitleskip}{-1em} % Reduce space before title
\setlength{\cftaftertoctitleskip}{1.5em} % Space after title
\setlength{\cftbeforetoctitleskip}{50pt} % Space above TOC title (same as chapters)
\renewcommand{\cftchappagefont}{\normalfont} % Page numbers normal
\renewcommand{\cftchapdotsep}{\cftdotsep} % Add dot leaders
\setlength{\cftbeforesecskip}{0.5ex} % Reduce space between sections
% Add List of Figures/Tables to TOC
\renewcommand{\listfigurename}{List of Figures}
\renewcommand{\listtablename}{List of Tables}

\usepackage{pdfcomment}
\begin{document}

% Frontispiece
\begin{titlepage}
    \centering
    \includegraphics[width=0.2\textwidth]{images/unibo-logo.pdf}
    
    \vspace*{1cm}
    \Large
    \textbf{ALMA MATER STUDIORUM \\ 
    UNIVERSITÀ DI BOLOGNA}
    
    
    \vspace{1.5cm}
    
    \normalsize
    \textsc{Department of Classical Philology and Italian Studies} \\
    \vspace{0.5cm}
    \textsc{Second Cycle Degree in} \\
    \vspace{0.2cm}
    \textsc{\textbf{Digital Humanities and Digital Knowledge}}
    
    \vspace{1.6cm}
    
    \begin{spacing}{1}
    \LARGE
    \textbf{From Documents to Dialogue:\\Design, Implementation and Evaluation of a Question-Answering System for Geoportale Nazionale Archeologia}
    \end{spacing}
    
    \vspace{1.2cm}
    \normalsize
    Dissertation in\\
    \textbf{Machine Learning for the Arts and Humanities}
    
    \vspace{1cm}
    
    \begin{tabbing}
    \textbf{Supervisor} \hspace{10cm} \= \textbf{Defended by} \\
    Prof. Giovanni Colavizza \> Lucrezia Pograri \\
    \\
    \textbf{Co-Supervisors}\\
    Prof. Paolo Bonora \\
    Mario Caruso and Simone Persiani, BUP Solutions
    \end{tabbing}
    
    \vfill
    \rule{\linewidth}{0.4pt}
    \vspace{0.2cm}
    
    \textbf{Graduation Session II} \\
    \textbf{Academic Year 2024/2025}
    
\end{titlepage}
\newpage
\thispagestyle{empty}  % Force blank page to have no header/footer
\mbox{}                % Prevent "empty page" warning
\setcounter{figure}{0}

%\pagestyle{empty}
% Front Matter (Roman page numbering)
\frontmatter
\pagestyle{fancy}
\tableofcontents

% Abstract
\chapter{Abstract}
\label{chap:abstract}
\begin{spacing}{1.5}
At the confluence of artificial intelligence and digital humanities, this thesis explores the deployment of retrieval-augmented generation (RAG) to facilitate access to the \textit{Geoportale Nazionale Archeologia (GNA)}, the Italian central repository of archaeological data. The study presents the design, implementation, and assessment of a dedicated question-answering system which integrates semantic embeddings, hybrid retrieval mechanisms, transformer-based language models, and user feedback loops into a modular pipeline.

Evaluation combined quantitative benchmarking with qualitative analysis by expert users, yielding results that underscore both the promise and the fragility of RAG in a cultural heritage context. The system achieved marked improvements in the retrieval of procedural guidelines and technical reports, accompanied by a reduction in misleading or extraneous information. Nonetheless, experiments revealed sensitivities to document structure and inconsistencies in provenance tracking, together with the challenge of balancing computational efficiency against contextual fidelity.

Far from claiming semantic comprehension, the system positions itself as a mediating tool that orients archaeologists and heritage professionals within vast textual corpora, surfacing relevant passages and easing interpretive navigation. Beyond the archaeological domain, its significance emerges in demonstrating how technical innovation intersects with infrastructural limitations and ethical imperatives, thereby situating AI not as a surrogate for scholarly judgment but as an instrument capable of extending humanistic inquiry and fostering renewed modes of interpretation and engagement across the digital humanities.

\vspace{\baselineskip} % Add vertical space before keywords
\noindent\textbf{Keywords:} Digital Humanities \textperiodcentered\ Information Retrieval \textperiodcentered\ Question-Answering Systems \textperiodcentered\ Retrieval-Augmented Generation \textperiodcentered\ Machine Learning \textperiodcentered\ Natural Language Processing \textperiodcentered\ Humanistic AI \textperiodcentered\ Cultural Heritage.

\end{spacing}
\clearpage
\listoffigures
\clearpage
\listoftables

\newpage
\thispagestyle{empty} 
\mbox{}                
\setcounter{figure}{0}

% Main Content (Arabic page numbering)
\mainmatter
\pagestyle{fancy}
\renewcommand{\figureautorefname}{Fig.}
\renewcommand{\tableautorefname}{Tab.}
\renewcommand{\chapterautorefname}{Chap.}
\renewcommand{\sectionautorefname}{§}
\makeatletter
\renewcommand{\thefootnote}{\textsuperscript{\arabic{footnote}}}
\renewcommand\@makefntext[1]{%
  \noindent
  \parindent=0pt
  \leftskip=0pt
  \hb@xt@1.8em{\hss\@thefnmark}#1%
}
\makeatother
\chapter{Introduction}
\label{chap:introduction}
\begin{spacing}{1.5}  % line spacing
At the swiftly evolving intersection of artificial intelligence (AI) and digital humanities (DH), computational methods have profoundly transformed access to and interpretation of cultural heritage resources. Among these, question-answering systems (QASs) -- driven by advances in natural language processing (NLP) and retrieval-augmented generation (RAG) -- have become increasingly significant tools, offering new possibilities of engaging with extensive documentation and complex repositories. This thesis arises directly from an applied research experience conducted during an internship at \href{https://www.bupsolutions.com/en/home_en/}{BUP Solutions}\nocite{bup_solutions_bup_nodate}, aimed at exploring the realistic feasibility and effectiveness of AI technologies in the context of cultural heritage. Specifically, the project focused on the desing, implementation and evaluation of a specialised QAS for the \textit{Geoportale Nazionale Archeologia (GNA)}, Italy’s primary repository of archaeological data under the auspices of ministerial authorities. For clarity, throughout this work the implemented system will be referred to interchangeably as the ``GNA QA system'' or the ``GNA AI assistant''.

The motivation of the present study stemmed from a practical challenge: facilitating efficient, intuitive, and accurate access to the extensive and often fragmented body of archaeological documentation hosted by the GNA. Archaeologists, heritage professionals and scholars working with this resource frequently face difficulties in navigating vast volumes of intricate technical reports, field notes, procedural guidelines, and complex geospatial data. In response, the project experimented with applying cutting-edge NLP and machine learning (ML) techniques -- primarily transformer-based language models combined with advanced retrieval methods -- to dynamically locate and synthesise relevant information based on user queries expressed in natural language.

Central to the chosen methodology is RAG, an approach that significantly enhances traditional QASs through the dynamic retrieval of domain-specific content, which augments the generative capabilities of language models. Instead of relying solely on internal model knowledge, systems grounded in RAG integrate external document retrieval with generative text production, resulting in greater reliability and outputs tethered in evidentiary contextual material -- crucial qualities for scholarly and professional uses. While this approach inherently promises increased accuracy and reduced hallucinations compared to purely generative methods, it also involves several complexities and uncertainties, which were encountered firsthand during the development and evaluation phases, as will be discussed in the following chapters.

Rather than adopting a narrowly theoretical or idealised perspective, this study reflects the exploratory and evolving nature of hands-on experimentation, shaped by iterative cycles of trial-and-error, heuristic adjustments, and pragmatic resolutions to practical constraints such as computational limits, the absence of standardised evaluation benchmarks, and the structural complexity of the domain. This process brought to light the persistent tension between the ambitions of AI solutions and the realities of applying them in intricate cultural contexts. In systems like the GNA’s AI assistant, the focus necessarily shifts from abstract notions of understanding to measurable outcomes: the true test is not whether the system comprehends archaeology in any human sense, but whether it efficiently retrieves relevant information, handles the complexities of the domain, and supports users in making informed decisions. Against such backdrop, one might ask: how far can technical ingenuity propel us before we run up against the unique subtleties of human knowledge and practice? Here, McDermott’s essay \textit{Artificial Intelligence Meets Natural Stupidity} offers a timely reminder, warning against the lure of \textit{wishful mnemonics} in AI and urging us to resist this inflationary language and the temptation to label what our systems do with grand terms like ``understand''. Instead, McDermott advocates for a clear-eyed assessment and communication of what these systems actually accomplish -- and an equally frank acknowledgment of where their true limits lie. Only through such intellectual honesty can the field avoid self-delusion and maintain its credibility \citep{mcdermott_artificial_1976}.

In light of this reality, this study deliberately avoids overstating the system’s semantic or interpretive capabilities. Instead, it foregrounds the project’s exploratory nature, acknowledging both methodological achievements and encountered limitations. The outcome represents a pragmatic effort toward applying AI in the digital humanities, offering insights into the real-world challenges and possibilities of using retrieval-augmented generation in cultural heritage contexts.

This work remains, at its heart, fundamentally hopeful. It demonstrates that even in the face of inherent methodological challenges, AI-driven tools carry genuine promise for enhancing access to cultural heritage resources. By presenting both the achievements and the limitations encountered along the way with transparency, this thesis seeks to contribute to the ongoing dialogue between AI and the humanities, offering a vision of AI’s evolving role as a catalyst for new forms of stewardship, interpretation, and engagement with cultural heritage.

\end{spacing}
\chapter{The Evolution of Question-Answering Systems}
\label{chap:QAS}
\sloppy
\begin{spacing}{1.5}

This chapter introduces the foundations of question answering (QA) as both a computer science discipline and an applied task. Before the emergence of large language models (LLMs),\footnote{Large Language Models (LLMs) are advanced AI systems trained on massive text datasets to generate and understand human language. For an accessible overview, see \href{https://mark-riedl.medium.com/a-very-gentle-introduction-to-large-language-models-without-the-hype-5f67941fa59e}{\textit{A Very Gentle Introduction to Large Language Models without the Hype}} \citep{riedl_very_2023}.} Transformers,\footnote{The Transformer is a neural network architecture introduced in 2017 that efficiently models sequential data using a self-attention mechanism. The original paper, \textit{Attention Is All You Need} by Vaswani et al. (\citeyear{vaswani_attention_2017}), provides a foundational outline.} and modern generative AI,\footnote{Generative AI refers to systems capable of producing new content, such as text, images, or audio, based on learned patterns. For more, see the \textit{Stanford AI Index 2025 Report} \citep{maslej_artificial_2025}.} question-answering systems (QAS) progressed through distinct paradigms: from symbolic and rule-based architectures to classic information retrieval (IR) models and early neural networks approaches \citep{jurafsky_chapter_2024,antoniou_survey_2022}. Early systems depended on domain-specific adaptations, manually curated knowledge bases, keyword retrieval, and engineered features. In recent years, transformer-based language models such as BERT and GPT have significantly advanced the capabilities of QA systems by enabling both answer extraction and text generation. Unlike their predecessors, these models can generate or extract responses using deep contextual understanding derived from large-scale pretraining \citep{kaplan_scaling_2020}. However, they tend to exhibit factual inaccuracies, shallow contextual understanding in certain scenarios, and limited adaptability to new or evolving information. They also frequently hallucinate or generate outdated responses, constrained by their static training corpora \citep{harsh_comprehending_2024}.

\section{Pre-Transformer Era: Symbolic and Statistical Systems}
The development of QAS prior to the rise of Transformers was shaped by several key methodological shifts and technological milestones. These earliest efforts prioritised manually curated knowledge bases and rules-based systems for precise but limited question matching. As the scope of QA expanded, techniques evolved to incorporate large-scale information retrieval methods, statistical modeling, and increasingly complex approaches to feature engineering and answer extraction. This direction ultimately set the stage for early neural models that leveraged word embeddings and sequence modeling, gradually moving the discipline toward data-driven architectures and deeper semantic representation.

\subsection{Rule-Based Systems (1960s--1980s)}
Early QAS relied on highly constrained, domain-specific approaches built around manually constructed knowledge bases. These systems operated within carefully delineated boundaries, matching user questions to a limited set of predefined templates and answer patterns. While this design enabled highly precise responses in their target domains, it also rendered the systems brittle and inflexible -- minor variations in user queries or topics outside the encoded scope often resulted in failure to provide meaningful answers.

Expert systems from this era encoded explicit inference rules and logical representations of knowledge, enabling a form of automated reasoning that was fundamentally deterministic. However, these approaches struggled to address ambiguity or generalise beyond the hand-curated domain, and could not scale to larger, more dynamic information environments \citep{noauthor_question_2025, jurafsky_chapter_2024}.

Seminal examples of early domain-specific QA systems include:
\begin{itemize}
    \item \textbf{BASEBALL} (1960s): hand-coded rules and database logic for Major League Baseball\footnote{Major League Baseball (MLB) is the leading professional baseball league in North America. It is regarded as the world’s premier baseball competition.} questions \citep{green_baseball_1961};
    \item \textbf{SHRDLU}\footnote{SHRDLU was developed at the MIT Computer Science and Artificial Intelligence Laboratory (CSAIL) between 1968--70. The software allowed users to interact conversationally with a program that could manipulate, describe, and answer questions about objects in a virtual \``blocks world\'', a simplified environment containing various movable blocks. Read more about SHRDLU program here: \url{https://hci.stanford.edu/winograd/shrdlu/}.} (late 1960s): symbolic reasoning for a blocks-world robot in a toy domain \textcolor{blue}{(Winograd, 1971)};
    \item \textbf{LUNAR} (1971): pattern matching and restricted knowledge base for geological questions about Moon rocks \citep{woods_lunar_1972};
    \item \textbf{Unix Consultant (UC)}\footnote{UC (QA) system, created at U.C. Berkeley (CA), answered queries about the Unix operating system using a hand-crafted knowledge base and could tailor responses to different user types \citep{robert_berkeley_1988}.}  and \textbf{LILOG}\footnote{LILOG project was as a text-understanding system designed for tourism information in a German city \citep{noauthor_question_2025}.} (1980s): domain-specific QA via linguistic rules and expert knowledge; though both projects remained at the demonstration stage, they contributed to advancing research in computational linguistics.
\end{itemize}

These early QA systems demonstrated the potential of automated question answering but highlighted the central challenge of balancing precision with generality and scalability. Their evolution would motivate the subsequent shift toward statistical and data-driven approaches \citep{jurafsky_chapter_2024, antoniou_survey_2022}.

\subsection{Classic Information Retrieval Strategies (1990s--mid-2010s)}
As the volume of unstructured web data grew, QA moved toward ranking text passages with IR techniques like TF-IDF\footnote{TF-IDF (Term Frequency-Inverse Document Frequency) is a statistical method for ranking how important a word is to a document in a collection.} and BM25,\footnote{BM25 is a ranking function that improves information retrieval by considering term frequency, document length, and saturation effects.\\For more details on TF-IDF and BM25, read \textit{Introduction to Information Retrieval} \citep{manning_introduction_2008}.} to locate relevant content within large text collections. Open-domain QA systems -- such as those in TREC QA\footnote{TREC QA refers to the Question Answering track of the Text REtrieval Conference (TREC), a long-running evaluation series that has set benchmarks for open-domain QA research since 1999. See \url{https://trec.nist.gov/data/qa.html}\notecite{noauthor_text_nodate}} \citep{hirschman_natural_2001} -- shifted the focus from structured fact retrieval to returning ranked sentences or extracting answer spans from retrieved passages. These approaches made it possible to scale QA to a broad range of topics and data sources, yet they also introduced notable challenges. Lacking deep understanding of natural language, IR-based QA systems often failed to interpret nuances, synonyms, or complex phrasing, and frequently missed correct answers that did not explicitly match the user’s query terms \citep{antoniou_survey_2022, caballero_brief_2021}.

\subsection{Statistical Models and Feature Engineering (2000s--2018)}
During the 2000s and early 2010s, QA began to move beyond brittle rule-based systems. Instead of relying on hand-crafted heuristics alone, researchers increasingly turned to statistical methods capable of reasoning over large corpora. N-gram models and statistical IR techniques -- e.g., TF-IDF, BM25 and probabilistic models\footnote{Language Models for IR (LMIR) -- such as n-gram models -- estimate the probability of a query being generated by a document's language model. They capture local word dependencies and were widely used in early QA, speech recognition, and spelling correction \citep{ponte_language_1998}, but were later outperformed by models like RNNs, LSTMs, and Transformers due to their limited handling of long-range context} -- provided the first real capacity to navigate and rank massive text collections with some measure of relevance. By weighting terms according to their frequency and informativeness, these models made it possible to automatically surface candidate passages from unstructured data, a crucial step in scaling QA systems to the size of the web, large repositories and archives \citep{manning_introduction_2008}.

A major milestone of this period was IBM's \textit{Watson} system, which achieved notable success by winning the \textit{Jeopardy!} quiz competition in 2011.\footnote{The ``Jeopardy Challenge'' was a high-profile test where IBM \textit{Watson} competed on the American television quiz show \textit{Jeopardy!} against two of the show's greatest human champions. Watson’s victory demonstrated significant progress in machine comprehension and open-domain question answering (\href{https://en.wikipedia.org/w/index.php?title=IBM_Watson&oldid=1301611671}{Wikipedia IBM Watson}). In February 2013, IBM announced that \textit{Watson}'s first commercial deployment would assist with utilization management decisions for lung cancer treatment at Memorial Sloan Kettering Cancer Center in New York City, in partnership with WellPoint (now Elevance Health) \citep{upbin_ibms_2013}.} Watson’s \textit{DeepQA} architecture integrated hundreds of NLP, IR and ranking components, employing sophisticated pipelines to analyse and combine evidence from diverse sources \citep{ferrucci_building_2011}. However, despite its advanced design, \textit{Watson} relied on non-generative methods; it synthesised and ranked candidate answers but did not generate free-form responses from scratch.

Simultaneously, semantic QA systems also matured, mapping natural language (NL) questions to structured queries  -- e.g., using SPARQL -- executed over knowledge bases like Freebase and DBpedia. These systems required advanced components for entity recognition, relation extraction, and reasoning over symbolic representations. Typical architectures included steps like question analysis, sentence mapping, disambiguation, and query building, enabling automatic translation of NL into formal queries over RDF data sources. Thanks to the usage of ontology-mapping and linguistic resources -- e.g., WordNet \citep{miller_wordnet_1992} and BabelNet \citep{navigli_ten_2021} --, these approaches further bridged the gap between unstructured text and structured knowledge bases \citep{franco_ontology-based_2020}.

Throughout this period, feature engineering was the beating heart of QA. Techniques such as conditional random fields (CRFs) and support vector machines (SVMs) enabled models to exploit hand-crafted features -- including lexical overlap, question type, and answer patterns -- to enhance answer extraction from retrieved texts. Hybrid QA systems also appeared, combining keywords-based IR methods for unstructured sources with knowledge-base querying for fact-based answers, thereby improving both coverage and precision \citep{antoniou_survey_2022}.

This period, although still extractive and feature-dependent, set the stage for what followed. It demonstrated that scaling QA required both statistical reasoning over large corpora and semantic mapping into structured resources. All the while, it highlighted the bottlenecks of hand-engineered systems: they were labour-intensive to build, brittle across domains, and ultimately limited in their ability to capture deeper semantic relations. The gradual introduction of distributed word representations toward the end of this period hinted at a new trail, one that would come fully into focus with the neural architectures of the late 2010s.

\subsection{Early Neural and Generative Models (Late 2010s)}
The late 2010s marked a profound transition, as QA systems began to absorb the lessons of neural representation learning. The introduction of distributed word embeddings -- Word2Vec \citep{mikolov_efficient_2013}, GloVe \citep{pennington_glove_2014} and similar models -- shifted the paradigm from sparse statistical features to dense, continuous vector spaces. Instead of simply counting word overlaps, systems could now measure semantic proximity between terms, enabling them to recognise that, for example, ``excavation'' and ``dig'' refer to related concepts. This advance laid the groundwork for capturing meaning beyond surface forms, improving both retrieval and answer matching \citep{jurafsky_chapter_2024}.

Embedding representations enabled the rise of recurrent architectures, particularly recurrent neural networks (RNNs), long short-term memory (LSTM) networks, and gated recurrent units (GRUs), which for the first time allowed systems to process language as sequences rather than bags of words. These models could in principle carry information across multiple tokens, making them attractive for reading comprehension tasks where the relation between question and passage unfolds over several sentences. Benchmarking datasets such as SQuAD \citep{rajpurkar_squad_2016} and Natural Questions \citep{kwiatkowski_natural_2019} became testing grounds for these methods, with LSTM-based encoders achieving state-of-the-art results by aligning question and context representations. Yet, the limitations quickly became evident: RNNs were notoriously poor at handling long-range dependencies, leading to failures when reasoning was required across multiple sentences, paragraphs, or documents \citep{jurafsky_chapter_2024}.

Around this time, researchers also began to experiment with generative models for QA, drawing inspiration from machine translation. Encoder-decoder architectures offered the tantalising possibility of producing answers as free-form text instead of merely extracting spans from source documents. These early generative QA systems demonstrated that models could synthesise responses in natural language, opening the door to more conversational applications. However, their outputs were often unreliable. Many simply rephrased the input passage, hallucinated details not grounded in evidence, or failed to maintain coherence when stitching together information from multiple contexts \citep{caballero_brief_2021}.

These developments set the stage for the subsequent breakthroughs brought about by attention mechanisms and transformer-based architectures, which dramatically improved the handling of context and factuality in generative QA.

\section{Blind Spots and Bottlenecks: The Shortcomings of Early Approaches}
Earlier approaches to question answering were hindered by several fundamental limitations. Most notably, symbolic and rule-based systems suffered from severe domain restrictions, as their performance relied on hand-crafted knowledge bases and rigid rules that did not generalise well to new or broader topics \citep{alqifari_question_2019}. The brittleness of these systems was further exposed by their heavy dependence on template matching, which frequently led to failures when users phrased questions in unanticipated ways or employed linguistic variations \citep{hirschman_natural_2001}. Statistical and IR models, while more scalable, continued to struggle with true semantic understanding and contextual reasoning, often retrieving only superficially relevant snippets in place of synthesising comprehensive or contextually rich answers \citep{alanazi_question_2021, diefenbach_core_2018}. The answers these systems produced were typically shallow, extracted verbatim from source texts rather than generated or adapted to the user’s specific information need \citep{hirschman_natural_2001,alqifari_question_2019}.

Substantial manual effort was required to design, maintain, and update rules, features, and parsers, creating significant bottlenecks and making adaptation to new domains costly and time-consuming \citep{alanazi_question_2021}. In addition, IR and knowledge base (KB) approaches frequently exhibited incomplete coverage, missing relevant answers due to differences in phrasing or limitations in their underlying datasets \citep{diefenbach_core_2018}. Early neural models, despite improvements, were generally confined to handling short text spans and struggled with complex or multi-sentence reasoning tasks. Finally, all these methods exhibited a strong dependence on the quantity and quality of available training data and engineered features, resulting in inconsistent performance across different domains and question types \citep{liu_challenges_2022,alanazi_question_2021,alqifari_question_2019,diefenbach_core_2018,hirschman_natural_2001}.

These cumulative factors left pre-generative systems largely inflexible and frail for QA purposes, with limited ability to provide context-aware, nuanced, or creative responses to user queries.

\section{Deep Learning Breakthroughs}
The advent of the Transformer architecture fundamentally reshaped the field of deep learning and revolutionised neural QA. Introduced by \citeauthor{vaswani_attention_2017} in 2017, Transformers replaced RNNs and LSTMs with a self-attention mechanism that could model relationships between words regardless of their distance in the input sequence. This innovation allowed for efficient parallelization during training and inference, greatly improving the scalability and performance of language models on a range of NLP tasks, including QA.

One of the earliest and most influential transformer-based models was BERT (Bidirectional Encoder Representations from Transformers) \citep{devlin_bert_2019}. BERT employs a bidirectional attention mechanism and is pretrained using a masked language modeling objective, allowing it to capture complex context from both directions in a sentence. When fine-tuned for QA benchmarks -- e.g., SQuAD --, BERT achieved unprecedented accuracy, reaching Exact Match and F1 scores above 85\% and 87\% respectively on the SQuAD 2.0 leaderboard, thus surpassing previous neural models and establishing a new standard for QA \citep{li_death_2024}.

Building on this foundation, subsequent models explored variations and enhancements of the Transformer paradigm. XLNet, for example, employed a permutation-based language modeling objective, enabling it to better capture bidirectional context and achieve state-of-the-art results on several QA benchmarks \citep{yang_xlnet_2020}. In specialised domains, models such as BioBERT extended the BERT architecture with additional pretraining on biomedical texts, achieving top performance on domain-specific challenges like the BioASQ competition \citep{yoon_pre-trained_2019}. Parallel research into model architectures also produced frameworks such as Dynamic Coattention Networks (DCN), which fused question and context representations through attention mechanisms and iterative decoding, further improving accuracy on reading comprehension tasks \citep{xiong_dynamic_2018}.

These breakthroughs ushered in a new research culture. QASs became systematically optimised at every stage, from tokenisation and embedding to retrieval and answer extraction \citep{farea_understanding_2025}. At the same time, the flexibility of Transformers encouraged exploration into conversational QA, multi-turn dialogue, and domain-specific fine-tuning \citep{yue_survey_2025}. Yet, for all their impact, transformer-based models still relied primarily on \textit{parametric memory}: their knowledge remained bounded by the data seen during pretraining. This limitation set the stage for a new class of approaches designed to bridge the gap between the static nature of models and the dynamic, real-world information needs.

\section{Large Language Models, Agents and Modular Pipelines}\label{sec:llm-agents}
Currently, a clear distinction emerges between ``traditional'' QA systems, primarily built upon general-purpose pretrained language models, and the new wave of modular approaches that dynamically retrieve external information sources. Traditional QA encompasses both extractive and generative paradigms, each defined by how they use the model’s internal knowledge. Extractive QA models are designed to identify and extract exact spans directly from a provided text or document, making them highly effective for fact-based questions and reading comprehension tasks. Generative QA models, in contrast, use natural language generation (NLG) to produce answers, typically synthesising or paraphrasing responses in ways that may not appear verbatim in the original text. However, despite their success, both of these paradigms are fundamentally limited by the static nature of their training data. They may struggle with rare, fast-changing, or domain-specific queries, and are prone to hallucinations\footnote{In the context of LLMs, hallucinations refer to outputs that are plausible-sounding but factually incorrect, fabricated, or unsupported by the underlying data or external sources \citep{harsh_comprehending_2024}.} and outdated information \citep{farea_understanding_2025}.

Recent advances in question answering are characterised by the emergence of retrieval-augmented generation (RAG). In these pipelines, a retriever component dynamically accesses external knowledge bases, while a generator conditions on the retrieved information to produce grounded answers. This approach addresses many of the shortcomings of earlier transformer-based models and significantly enhances factual accuracy, contextual relevance, and system adaptability. Generative LLMs within the RAG pipeline are able to incorporate real-time knowledge, thereby reducing hallucinated content and providing up-to-date responses, even as external data sources evolve \citep{yue_survey_2025,lewis_retrieval-augmented_2020}. Benchmarks show that RAG-enhanced models significantly outperform standard LLMs in factual QA, particularly in domains demanding precise recall or up-to-date knowledge. For instance, enterprise evaluations demonstrate up to 30-40\% improvement in incorporating domain-specific terminology compared to standalone models, while user trust increases substantially when source citations are included \citep{vaibhav_retrieval-augmented_2025}.

Furthermore, RAG-based QA systems offer practical advantages for scalability. Rather than requiring full model re-training to accommodate new information, they can simply update or expand the external KB. This design allows for the integration of vast and dynamic data resources, enabling high coverage across domains and rapid adaptation to new information needs. However, these benefits come with trade-offs. RAG architectures require more complex infrastructures, including document indexing and retrieval mechanisms, which increase operational overhead and latency compared to traditional, static QA systems. As a result, deploying and maintaining RAG-based systems can be more challenging, especially at scale.

Beyond RAG, a parallel evolution is visible in the emergence of LLM-based agents. Unlike monolithic models, these agents operate as orchestrators of multi-stage reasoning, combining planning, question understanding, retrieval, reasoning, and answer generation in an iterative loop. Architectures typically integrate memory to retain conversational context, planning modules to decide on next actions, and reasoning modules to balance internal ``thinking'' with external interactions, such as calling APIs, querying databases, or consulting humans \parencite{yue_survey_2025}. It overcomes the rigidity of earlier pipelines, which relied on static submodules trained in isolation, and the limitations of naive LLM QA, which lacks external grounding and dynamic adaptability. Agents thus introduce a form of controlled autonomy: they not only retrieve information but also decide \textit{when} and \textit{how} to engage tools, creating more flexible and resilient QA systems.

Current research highlights that these modular, agentic pipelines offer more than incremental improvements. They introduce transparency through source attribution, factual grounding, and explainability -- qualities increasingly demanded in high-stakes domains such as law, medicine, and cultural heritage. At the same time, promising directions include multimodal modes -- retrieving from text, images, or audio --, though cross-modal alignment remains an open challenge \citep{vaibhav_retrieval-augmented_2025}; hybrid retrieval that combines sparse lexical methods with dense search; and adaptive systems that dynamically tune retrieval and reasoning strategies based on query type and context (\cite{yue_survey_2025,vaibhav_retrieval-augmented_2025}). Taken together, these advances point toward a decisive shift: from static models locked within their parametric memory to dynamic, agentic systems capable of interacting with and reasoning over the evolving universe of human knowledge.

\autoref{tab:qa-comparison} summarises the functional differences between traditional and RAG-based QA systems, highlighting the shift toward dynamic, retrieval-augmented, and generative approaches that characterise the current state of the discipline. 

\addtocounter{table}{-1}
\begin{table}[H]
\centering
\begin{tabularx}{\textwidth}{l>{\raggedright\arraybackslash}X>{\raggedright\arraybackslash}X}
\toprule
\textbf{Feature} & \textbf{Traditional QAS} & \textbf{RAG QAS} \\
& \textit{(e.g., BERT, GPT-2/3)} & \textit{(Retriever + Generator)} \\
\midrule
Knowledge source & Fixed (training data) & Dynamic (external docs/databases) \\
Answer type & Extracted or generated & Retrieved + generated (synthesised) \\
Factual accuracy & Limited (can hallucinate or be outdated) & High (grounded in retrieved, up-to-date information) \\
Contextual depth & Limited & Comprehensive, nuanced \\
Scalability & Moderate & High (can update external data sources) \\
Computational cost & Lower & Higher (due to retrieval/generation) \\
Latency & Lower (faster for simple queries) & Higher (retrieval step adds time) \\
Complexity of setup & Simpler & More complex to maintain \\
Adaptability & Less adaptable to new domains & Highly adaptable via updated document index \\
\bottomrule
\end{tabularx}
\vspace{0.5em}
\caption{Comparison of traditional vs. retrieval-augmented generation question-answering systems.\\ \footnotesize{Adapted from \url{https://www.geeksforgeeks.org/nlp/rag-vs-traditional-qa/}\nocite{noauthor_rag_2025}}}.
\label{tab:qa-comparison}
\end{table}


The main stages in the evolution of QA systems, along with representative approaches and landmark examples, are summarised in \autoref{tab:qa_evolution}.

\addtocounter{table}{-1}
\begin{table}[H]
\centering
\begin{tabularx}{\textwidth}{>{\raggedright\arraybackslash\bfseries}X >{\raggedright\arraybackslash}X >{\raggedright\arraybackslash}X}
\toprule
\textbf{Models} & \textbf{QA Approach} & \textbf{Examples / Results}\\
\midrule
Symbolic / Rule-based (1960s–1980s) & Rule-based, domain-specific, handcrafted knowledge base & BASEBALL, LUNAR, SHRDLU \\
Early IR Approaches (1990s–mid-2010s) & Keyword retrieval, TF-IDF, BM25, open-domain ranking & TREC QA \\
Statistical / Seq2Seq (2000s–2018) & N-gram, embeddings, RNN/LSTM, statistical IR & Early neural QA, Reading comprehension in 2010s \\
Transformer-based & Pretraining, fine-tuning, self-attention & BERT (93\% F1 on SQuAD), XLNet \\
Generative LLMs and agents & Prompting, retrieval-augmented generation, agentic reasoning & GPT-3, RAG pipelines \\
\bottomrule
\end{tabularx}
\vspace{0.5em}
\caption{Evolution of question-answering systems}.
\label{tab:qa_evolution}
\end{table}

\end{spacing}

\chapter{State of the Art}
\label{chap:sota}
\begin{spacing}{1.5}
\sloppy
In the rapidly evolving landscape of artificial intelligence (AI), large language models (LLMs) have demonstrated remarkable results in text generation and understanding. Yet, when applied to real-world tasks such as question answering, these models still face significant limitations. As detailed in the previous chapter, LLMs are prone to hallucinations\footnote{In the context of LLMs, hallucinations refer to outputs that are plausible-sounding but factually incorrect, fabricated, or unsupported by the underlying data or external sources.}, rely on static and often outdated training data, and offer limited transparency or traceability in their outputs. Additionally, they may struggle to incorporate domain-specific context or organizational knowledge \citep{vaibhav_retrieval-augmented_2025}. These factors pose challenges for domains, like cultural heritage and archaeology, where reliability, provenance, and interpretive rigor are fundamental requirements.

To address these concerns, retrieval-augmented generation (RAG)\footnote{For more information about RAG technique, see \url{https://en.wikipedia.org/wiki/Retrieval-augmented_generation}.} has emerged as a crucial methodological advance. It improves the factual grounding and contextual relevance of generated answers, thorugh the integration of external and verifiable knowledge at inference time, thereby reducing the risk of generating fabricated or distorted information \citep{martineau_what_2023}. As discussed in \autoref{chap:QAS}, this approach represents a significant step beyond both traditional information retrieval and earlier neural QA models, which were often brittle, domain-dependent, or struggled to adapt to evolving information needs.
\sloppy
The adoption of RAG in question-answering reflects a broader evolution within the field: from early symbolic and rule-based systems, through statistical and information retrieval approaches, to today’s transformer-based, generative architectures. This shift has transformed not only the technical capabilities of QA systems but also their applicability to complex, heterogeneous knowledge domains.
\sloppy
Although initially developed for open-domain question answering and enterprise search \parencite{akkiraju_facts_2024, jiang_towards_2024, packowski_optimizing_2024, yang_ragva_2025, zhou_enabling_2025}, RAG pipelines are increasingly adopted in the humanities and cultural heritage contexts. In these sectors, where interpretive rigor, provenance, and information reliability are critical, RAG-based tools support scholars and professionals in navigating vast, fragmented knowledge repositories. While some initiatives employ RAG to analyze sensitive historical materials \citep{callaghan_prototyping_2025, ciletti_retrieval-augmented_2025, sergeev_talking_2025, fan_research_2025}, this thesis explores a distinct application: improving access to procedural and technical documentation, where clarity, consistency, and actionable guidance are the primary objectives.
\\

This chapter therefore provides a comprehensive overview of the state of the art in retrieval-augmented generation, situates RAG within the current research landscape, outlines its core mechanisms, and examines its recent application in the digital humanities.

\section{Core Mechanisms and Foundations of Retrieval-Augmented Generation}\setlength{\parskip}
{0pt}
Retrieval-augmented generation (RAG) is a hybrid approach that addresses key limitations of traditional LLMs, knowledge staleness, limited context awareness, and lack of output traceability \parencite{vaibhav_retrieval-augmented_2025,gao_retrieval-augmented_2024, gupta_comprehensive_2024}. While LLMs excel in generating fluent, human-like text, they often falter when facing domain-specific queries or requests for information beyond their training cutoff. RAG directly addresses these challenges by integrating external information retrieval within the generation process, ensuring outputs are more factual, current, and grounded in verifiable sources \citep{wang_searching_2024}.

\subsection{RAG Pipeline: Components and Standard Practices}
Modern RAG systems follow a multi-stage pipeline that has become increasingly modular and sophisticated in recent research \citep{vaibhav_retrieval-augmented_2025,wang_searching_2024}. 

The standard workflow includes the following components:
\begin{itemize}
  \item \textbf{Query understanding and classification:} Not all queries require retrieval from external sources. Advanced systems first analyse and classify incoming queries to determine whether retrieval is necessary or if the LLM alone suffices. This step leverages natural language understanding techniques to extract key entities, relationships, and user intent, improving efficiency and reducing unnecessary retrieval latency.
  %Data Preparation and Indexing
    \item \textbf{Document indexing and chunking:} Raw data from source documents is preprocessed: cleaned, segmented into manageable "chunks" at token, sentence, or semantic level, and converted into dense vector representations (embeddings). Recent studies recommend dynamic or semantic chunking over simple fixed-size splitting, as it better preserves context and improves retrieval quality -- especially in heterogeneous domains.
    \item \textbf{Embedding and Vector Database:} Both document chunks and user queries are embedded into a shared vector space using models fine-tuned for semantic similarity (e.g., BAAI/bge, LLM-Embedder, intfloat/e5). These vectors are stored in efficient vector databases (e.g., Milvus, Faiss, Qdrant), selected based on scalability, indexing strategies, and support for hybrid (vector plus keyword) search capabilities.
    \item \textbf{Retrieval and query transformation:} Upon receiving a user query, the system encodes it into a vector and retrieves the top-k most relevant chunks from the indexed knowledge base (KB) using similarity search. Robustness is enhanced through hybrid retrieval, which combines dense (vector-based, e.g., DPR, Contriever) and sparse (lexical, e.g., BM25) methods. Advanced query transformation techniques -- including query rewriting, decomposition, or hypothetical document generation (e.g., HyDE)-- can further improve retrieval effectiveness.
    \item \textbf{Re-ranking:} Initially retrieved candidates are often re-ranked based on relevance to the original query, using additional models (DLM-based) -- e.g., cross-encoders like monoT5, monoBERT, or RankLLaMA, which jointly consider the query and each candidate -- or more sophisticated algorithms through heuristics. This contextualization ensures that the most pertinent information is prioritized for the generative model.\\See \autoref{tab:rerank_algorithms} for a summary on re-ranking techniques.
    \item \textbf{Repacking and summarization:} In some cases, retrieved passages may be reorganized or summarized to distill key information, especially when dealing with lengthy corpora. This step can involve extractive summarization or abstractive (e.g., with Pegasus or T5) techniques to condense information and fit within the context window of the generator model.
    \item \textbf{Generation:} The generative model -- usually a transformer-based LLM such as T5, BART, or GPT -- synthesizes a response conditioned on both the original query and the retrieved context, integrating intrinsic model knowledge with external evidence to produce a coherent, accurate, and contextually grounded answer.
    \item \textbf{Post-processing and output delivery with attribution:} Generated response may undergo post-processing to ensure coherence, factual accuracy, and adherence to any specific output formats or constraints. This can include additional validation steps, such as fact-checking against trusted sources or applying heuristics to ensure the response is grounded in the retrieved context. Additionally, RAG systems increasingly support source attribution—citing or linking to retrieved documents to enhance transparency, interpretability, and user trust.
\end{itemize}
\citep{vaibhav_retrieval-augmented_2025,wang_searching_2024,gupta_comprehensive_2024}.


\begin{table}[H]
\centering
\footnotesize
\begin{tabularx}{\textwidth}{l X}
\toprule
\textbf{Algorithm} & \textbf{Rationale} \\
\midrule
Cross-Encoders & Joint encoding of query and document for fine-grained relevance scoring. \\
\cmidrule(lr){1-2}
TILDE \citep{zhuang_tilde_2021} & Token-level likelihoods for queries across a collection, allowing fast re-ranking by summing the probabilities of query tokens given each candidate passage. \\
\cmidrule(lr){1-2}
Learning-to-Rank (LTR \citep{gupta_comprehensive_2024}) & Traditional machine learning ranking approaches: \textbf{a) Pointwise:} predicts relevance score for each document independently; \textbf{b) Pairwise:} compares pairs of documents to learn relative relevance; \textbf{c) Listwise:} considers the entire ranked list at once.\\
\cmidrule(lr){1-2}
Hybrid sparse + dense scoring & Blends scores from dense retrievers (semantic similarity -- e.g., DPR, Contriever) and sparse methods (lexical overlap -- e.g., BM25, TF-IDF) for robust ranking. Sometimes uses learnable weighting \citep{wang_searching_2024}. \\
\cmidrule(lr){1-2}
Graph-based \citep{han_retrieval-augmented_2025} & Constructs a graph of candidates (nodes) based on relationships (semantic, citation, or knowledge graph edges), then uses graph algorithms  (e.g., PageRank, label propagation) to identify central passages. \\
\cmidrule(lr){1-2}
Self-RAG (LLM-enhanced reranking) \citep{asai_self-rag_2023} & Uses LLMs directly to score or select the most relevant passages, sometimes via few-shot prompting or chain-of-thought reasoning. \\
\bottomrule
\end{tabularx}
\vspace{0.5em}
\caption{Algorithms for document re-ranking in RAG pipelines.}
\label{tab:rerank_algorithms}
\end{table}







The RAG pipeline typically consists of two main stages: \textbf{retrieval} and \textbf{generation} (\cite{odsc-community_retrieval-augmented_2024}). The process begins with indexing, where raw data is cleaned, extracted, segmented into manageable "chunks", and encoded into vector representations. These embeddings are then stored in a vector database to facilitate efficient similarity searches. Upon receiving a user query, the system encodes it into a vector and retrieves the top-k most relevant chunks from the indexed knowledge base. In the second stage, the retrieved documents are passed to a generative model, often built upon Transformer architectures \citep{vaswani_attention_2017}. This module synthesizes the original query with the retrieved information to formulate a well-informed, coherent, and contextually appropriate response \citep{arslan_survey_2024}.

This modular mechanism (\autoref{fig:rag}) overcomes the limitations of static model parameters by continuously incorporating domain-specific and updated information. Recent contributions have helped to formally systematize the RAG pipeline's, with frameworks delineating specific interdependent modules such as query classification, retrieval, re-ranking, and generation \parencite{wang_searching_2024}.

\vspace{0.5em}
\begin{figure}[H]
  \centering
  \includegraphics[width=\textwidth]{images/rag_workflow.jpg} 
  \caption{Typical RAG workflow.\\
  \footnotesize{Source: \url{https://aws.amazon.com/de/what-is/retrieval-augmented-generation/}.\nocite{noauthor_was_nodate}}}
  \label{fig:rag}
\end{figure}
\vspace{0.5em}

\noindent In summary, the synergistic merging of LLMs' intrinsic knowledge with dynamic external sources allows for continuous knowledge updates and integration of domain-specific information, significantly enhancing response quality, particularly in knowledge-intensive and evolving domains \parencite{wang_searching_2024, gao_retrieval-augmented_2024}.

\section{Emerging Applications and Use Cases}\label{sec:evol_qas}
RAG systems are increasingly deployed across diverse domains -- spanning academia, enterprise, and product environments -- to enhance data accessibility, support decision-making, and facilitate natural language interaction with complex knowledge bases. Recent surveys and empirical studies document a growing array of scholarly applications of RAG, including:
\begin{itemize}
    \item Automated literature review tools and citation management -- e.g., LitLLM; \citep{agarwal_litllm_2025}, KNIMEZoBot; \citep{alshammari_knimezobot_2023};
    \item Generation of summaries for large corpora of academic papers;
    \item Field-specific knowledge extraction, including biomedical and legal research support.
\end{itemize}

\noindent In one experiment, a RAG system was developed to assist data scientists through a combination of GROBID for structured bibliographic extraction, fine-tuned embeddings, semantic chunking, and an abstract-first retrieval strategy. The system's performance, assessed using the Retrieval-augmented generation Assessment System (RAGAS), demonstrated improved faithfulness and context relevance in response generation \citep{aytar_retrieval-augmented_2024}. A similar approach was explored in the context of academic library systems, where RAG was applied to improve contextual retrieval through semantic indexing of structured metadata (e.g., MARC/RDA standards) and multimodal resources. Additionally, the framework introduced conversational querying via a natural language interface, supporting complex interdisciplinary searches and significantly improving document discoverability by synthesizing citation-backed responses from diverse scholarly sources -- including journals, datasets, and videos. This solution also addressed challenges such as copyright compliance and ethical AI transparency \citep{bevara_prospects_2025}.  Collectively, these studies affirm RAG systems’ efficacy in alleviating information overload and improving research workflow discoverability.

In parallel, the work of \citep{soman_observations_2024} provides further critical insights into the design of RAG systems for domain-specific and technical content, closely aligning with the methodological framework adopted in the GNA question-answering system. Using IEEE telecommunications engineering corpora (i.e., wireless LAN specifications and battery glossaries) as testbeds, their analysis highlights key factors influencing retrieval quality, which include chunk size, sentence-level similarity, and the strategic placement of domain-specific terms. These aspects are similarly addressed in the GNA RAG pipeline \citep{pograri_question-answering_2025}, which applies customized tailored chunking, semantic preprocessing, and contextual embedding strategies. Both studies advocate for more nuanced, context-aware approaches to enhance precision in technical and highly structured domains.

Numerous recent graduate-level research projects have provided substantive input into the implementation and evaluation of RAG systems:
\begin{itemize}
    \item \textcite{antolini_experimental_2025} developed a custom RAG system for open-domain question answering using both traditional (BM25, PRF) and advanced retrieval strategies, integrated with local LLMs. A novel Parametric RAG (PRAG) approach was also explored, embedding context into model parameters for performance gains.
    \item \textcite{caramanna_progettazione_2024} investigated conversational agent architectures, comparing various LLM types and retrieval configurations.
    \item \textcite{florio_progettazione_2024} implemented a LangChain-based RAG chatbot for corporate documentation, evaluating multiple vector database technologies.
    \item \textcite{salcuni_utilizzo_2025} applied RAG to the medical domain, improving LLM responses in hypertension care. The study used RAGAS to assess quality and relevance, focusing on personalization and accuracy.
    \item \textcite{nicoletti_llms_2025} developed Essence Coach, a chatbot that integrates LLMs with the Essence software engineering standard. This system significantly outperformed generic LLMs like GPT-4o in domain-specific reasoning tasks.
\end{itemize}

\section{RAG in the Digital Humanities}
A growing body of research is exploring RAG applications within the digital humanities. One such example is the \textit{iREAL} project, which applied RAG to interpret archival records from Aboriginal schools in Australia, demonstrating a careful balance between cultural sensitivity and historical accuracy \citep{callaghan_prototyping_2025}. Another initiative, \textit{ValuesRAG}, focuses on cultural alignment in LLMs by integrating societal and demographic knowledge through retrieval-augmented contextual learning, experimenting with the \textit{World Values Survey} dataset \citep{seo_valuesrag_2025}. In another case, the \textit{Foggia Occupator Dataset} project applied a RAG model to post-WWII Italian periodicals, extracting information on political figures and stylistic traits \citep{ciletti_retrieval-augmented_2025}.

RAG methodologies are being adopted within the GLAM sector (Galleries, Libraries, Archives and Museums) as well. In archival contexts, a smart assistant developed for querying the \textit{Prozhito} digital archive of personal diaries combines text-to-SQL filtering, hybrid search, and automatic query reformulation, proving especially effective for historians and anthropologists without prior knowledge of database query languages  \citep{sergeev_talking_2025}. Meanwhile, in museum settings, a comparative evaluation of RAG systems versus large-context LLMs for answering multimodal questions about artworks demonstrated that the RAG approach offers superior precision and explainability \citep{ramos-varela_context_2025}.

Innovations in graph-based retrieval are also gaining momentum \citep{belagatti_enhance_2024}. Techniques combining structured supervision and chain-of-thought prompting have been used to map character relationships in early modern English historiography, thereby reducing the manual workload typically associated with historical data annotation \citep{fan_research_2025}. Related directions are being explored within cultural heritage institutions, as seen in the \textit{CAT-IA} initiative \citep{barbato_nasce_2025}, which integrates ArCo knowledge graph \citep{carriero_arco_2019} within a RAG system for provenance tracking, AI explainability (XAI), and structured metadata extraction. Designed to streamline and enrich user interactions with the General Catalogue of Cultural Heritage \textit{(Catalogo generale dei beni culturali)}, \textit{CAT-IA} marks a notable stride in applying advanced digital technologies to promote accessibility and valorization of cultural assets.

Finally, efforts to advance access to fragmented digital repositories -- such as web archives -- have increasingly adopted RAG methodologies. An illustrative bespoke prototype \citep{davis_unlocking_2025} transforms keyword-based search into semantically guided question answering, sharing architectural parallels with the GNA QA system presented in the context of this thesis. Both systems prioritize semantic retrieval over lexical matching using dense embeddings -- e.g., \textit{E5} variants \citep{wang_text_2024} -- to interpret queries in context, employ structured text processing pipelines to reduce noise in source materials, and apply optimized chunking strategies for retrieval accuracy. Crucially, these studies highlight RAG’s potential to transform scattered and heterogeneous resources -- whether web archives or catalographic procedures -- into coherent, accessible knowledge through context-aware synthesis.

\section{Future Directions}
Ongoing research is pushing the boundaries of RAG, exploring innovative directions such as synthetic corpus generation, autonomous agent behavior, and AI-driven creative processes. One promising approach involves using synthetic corpora to enhance the robustness of RAG systems, improving their ability to generalize in low-resource domains \citep{bor-woei_generative_2024}. RAG is also increasingly applied to automate literature reviews and research synthesis. Agentic AI, such as the \textit{PaperQA} system \citep{lala_paperqa_2023}, has demonstrated the capacity to systematically retrieve and summarize academic literature through recursive querying and decision-making agents, enabling rigorous yet scalable analysis that holds significant potential for humanities scholarship \citep{han_automating_2024}.

\subsubsection*{Generative AI and Large‑Scale Agentic QA}
Large Language Models (LLMs) and prompting
After GPT‑1 (2018) and GPT‑2/3, large-scale decoder‑only models shifted capabilities: GPT‑3 in 2020 introduced few‑shot prompting and instruction‑based QA without fine-tuning.

These systems often serve as QA agents—able to plan, retrieve, reason, and generate in one pipeline, especially in retrieval‑augmented generation (RAG) frameworks.

QA as an agentic pipeline
Recent architectures treat QA as multi-stage agents: question understanding, retrieval (possibly over the web or knowledge stores), planning, reasoning, and generative answer synthesis. These yield state-of-the-art results beyond extractive systems.
(Yue 2025)
---------------------

Another significant advancement lies in semantic alignment. The integration of ontologies and knowledge graphs into RAG systems has rapidly advanced both the capabilities and the trustworthiness of LLMs. Ontologies, as formal domain knowledge models, provide structured frameworks that enable precise retrieval, semantic coherence, and the inclusion of ethical dimensions in generative AI. Complementing this, knowledge graphs capture complex relationships and support context-aware multi-hop reasoning, improving accuracy, explainability, and cultural sensitivity of outputs. Current research and practical applications span a range of initiatives – from ontology-guided entity typing to the grounding of AI in ethical principles and industrial procedure knowledge, demonstrating that these semantic tools are essential for creating robust, context-aware, and transparent RAG systems, addressing challenges in fields as diverse as healthcare, engineering, scientific discovery, and enterprise knowledge management \citep{tiwari_ontorag_2025, ludwig_ontology-based_2025, bran_ontology-retrieval_2024, sharma_og-rag_2024, xiao_orag_2024, park_ontology-based_2024, debellis_integrating_2024} + franco et al. 2020.

These advancements highlight the critical importance of developing RAG systems that complement rather than replace human interpretive judgment. Emerging approaches such as graph-RAG, ontology-aware models, multimodal architectures, and hybrid retrieval mechanisms represent key technical directions that support humanistic inquiry while upholding the epistemic and ethical standards of the field. Ultimately, RAG not only offers improved access to information, but also invites a reimagining of the relationship between artificial intelligence and cultural knowledge production, fostering tools that augment – not displace – human creativity and understanding.
\end{spacing}
\chapter{Case Study: A Question-Answering System for GNA}
\label{chap:casestudy}
\sloppy
\begin{spacing}{1.5} 

\section{Geoportale Nazionale per l’Archeologia (GNA)}
Geoportale Nazionale per l'Archeologia (GNA) \citep{mic_mic_2019} serves as the central online hub for the collection, management, and dissemination of data generated by archaeological investigations carried out across Italy \citep{acconcia_pubblicazione_2023}. Developed under the auspices of the Ministry of Culture (MiC), the project's primary goal is the creation of a dynamic archaeological map of the national territory, which is easily updatable over time, openly accessible, and designed for reuse and integration across multiple institutional and disciplinary contexts \citep{falcone_dematerializzazione_2023}.

The inception of the GNA traces back to a 2014 \textit{Memorandum of Understanding} signed by the Ministero dei Beni e delle Attività Culturali e del Turismo (MiBACT) -- specifically the Segretariato Generale, the Direzione Generale per le Antichità (DG-Ant), and the Consiglio Nazionale delle Ricerche (CNR). This agreement laid the groundwork for a national platform dedicated to the safeguarding and enhancement of cultural heritage through integrated digital infrastructure. However, it was the establishment of the Istituto Centrale per l’Archeologia (ICA) in 2016 that provided the structural and institutional foundation for the GNA. The ICA’s mandate to define standards and promote digital archaeological databases gave renewed potential to the initiative, which culminated in the launch and formal presentation of the GNA at a ministerial venue in 2019 \citep{calandra_il_2023}.

Far from being a mere data aggregator, the GNA serves as a dynamic knowledge base, collecting digital contributions from professional archaeologists -- especially those active in preventive archaeology -- as well as from research groups, universities, and concession-holders. Its scope encompasses a wide spectrum of outputs, ranging from vector data based on QGIS\footnote{QGIS is a free, open-source Geographic Information System (GIS) software used for creating, managing, and analysing geospatial data.} to reports, documentation packages, and datasets from academic and research projects. Data publication within the GNA is managed with attention to quality standards, intellectual property rights, and open-access principles, supported by the assignment of DOIs and distribution under Creative Commons licensing (CC-BY 4.0), ensuring both traceability and reusability \citep{acconcia_pubblicazione_2023,falcone_dematerializzazione_2023,boi_il_2023}. The platform is also aligned with European and Italian open data and transparency regulations, fulfilling requirements of national FOIA provisions and EU directives\footnote{The FOIA (Freedom of Information Act) Guidelines are documents issued by the Italian National Anti-Corruption Authority (ANAC) to clarify and guide the implementation of the right to generalised civic access in Italy. The guidelines -- especially those from 2016 -- define the limits and exclusions to access, as well as specify the publication and transparency obligations for public administrations.\\Read more at \url{https://foia.gov.it/normativa}.\nocite{noauthor_normativa_2016}} \citep{falcone_dematerializzazione_2023}.

\subsection{Purpose and Scope}
As the official repository for all research activities in archaeology -- particularly those related to public infrastructure projects -- the GNA platform was established to provide a unified national access point to essential archaeological data gathered nationwide. This includes the interventions listed in \autoref{tab:gna_data_sources}, all conducted under the scientific supervision of the Italian Ministry of Culture (MiC) \citep{acconcia_pubblicazione_2023,falcone_dematerializzazione_2023}.

\addtocounter{table}{-1}
\begin{table}[H]
\centering
\footnotesize
\begin{tabularx}{\textwidth}{ l >{\justifying\noindent\arraybackslash}p{0.65\textwidth} }
\toprule
\textbf{Archaeological interventions} & \textbf{Description} \\
\midrule
Preventive archaeology reports & Data from excavations and surveys carried out ahead of construction projects (e.g., highways, railways, pipelines), often submitted by private firms or cultural heritage consultants. \\
\cmidrule(lr){1-2}
Assisted scientific excavations records & Results from academic digs by universities or research institutions, including documentation of stratigraphy, finds, and site interpretation. \\
\cmidrule(lr){1-2}
Accidental discoveries & Locations of fortuitous archaeological finds, such as during agricultural work or construction, reported to local heritage authorities. Typically include preliminary spatial data and descriptive reports. \\
\cmidrule(lr){1-2}
Scheduled excavations & Long-term planned investigations, often at known heritage sites, including geospatial boundaries, uncovered structures, and findings. \\
\cmidrule(lr){1-2}
Archaeological surveys & Surface survey data with GPS-tracked locations of finds, artifact scatters, and site features. \\
\cmidrule(lr){1-2}
Cultural heritage GIS layers & External datasets from institutions (regional superintendencies, local governments, ICCD), e.g., maps of protected zones, risk maps, or site inventories. \\
\cmidrule(lr){1-2}
Legacy data and digitised archives & Georeferenced digitizations of paper maps, notebooks, and archival records previously stored in non-digital formats, essential for integrating historical with current data. \\
\cmidrule(lr){1-2}
Depository locations & Georeferenced storage locations of archaeological finds (museums, storerooms) associated with sites or interventions. \\
\cmidrule(lr){1-2}
Remote sensing and aerial surveys & Drone imagery, LiDAR scans, or satellite data used to identify and map archaeological features not visible at ground level. \\
\cmidrule(lr){1-2}
Paleontological sites & A specific level dedicated to paleontological sites is currently under study for future inclusion, aiming to protect this fragile heritage. \\
\bottomrule
\end{tabularx}
\vspace{0.5em}
\caption{Types of archaeological data sources integrated into Geoportale Nazionale per l'Archeologia.}
\label{tab:gna_data_sources}
\end{table}

\noindent These sources, once georeferenced and structured, are integrated into the GNA using standardised metadata and visualisation protocols, to allow users to view, search, and analyse information in a spatially accurate and coherent manner \citep{boi_il_2023, acconcia_pubblicazione_2023}.

What makes this material especially demanding is the combination of heterogeneity in format, sources and regulatory frameworks. The GNA brings together preventive reports, excavation records, surveys, legacy archives, GIS layers, and remote sensing outputs, each with its own structure, level of detail, and standards of documentation. Their integration is further conditioned by legal and procedural frameworks, from FOIA transparency obligations to Creative Commons licensing and DOI assignment, which ensure accessibility but add additional layers of compliance. Equally significant are the operative guidelines and technical instructions that regulate how this information is produced, structured, and uploaded into the platform. For example, the MOPR (Modulo di Progetto) section provides step-by-step guidance for consultants preparing preventive archaeology reports, from the structuring of stratigraphic descriptions to the encoding of metadata fields \parencite{noauthor_compilare_2025}.\footnote{See: \textit{Compilare il MOPR}, \url{https://gna.cultura.gov.it/wiki/index.php/Compilare_il_MOPR}.} Similarly, technical notes \parencite{noauthor_brevi_2025} illustrate the proper use of QGIS software in generating and validating shapefiles prior to submission, underscoring the centrality of geospatial data in contemporary archaeological practice.\footnote{See: \textit{Brevi note su QGIS}, \url{https://gna.cultura.gov.it/wiki/index.php/Brevi_note_su_QGIS}.}

\subsection{Stakeholders and Intended Users}\label{sec:gna_plugin}
The development of the GNA saw significant acceleration during the COVID-19 pandemic, which provided both the urgency and institutional impetus toward the creation of a unified digital platform for managing archaeological data nationwide. This initiative built upon years of prior collaboration between key stakeholders, including the Istituto Centrale per l’Archeologia (ICA) and the Istituto Centrale per il Catalogo e la Documentazione (ICCD), who had already developed a cataloging structure to document archaeological assessments and identified sites within the Sistema Informativo Generale del Catalogo (SiGECweb) \citep{calandra_il_2023, boi_il_2023}. The pandemic underscored the limitations of purely textual cataloguing and sparked a shift toward a more dynamic and geospatially grounded approach, leading to the adoption of a GIS-based framework better suited for preventive archaeology and territorial planning. The result was a consolidated national infrastructure designed not only to support compliance with cultural heritage protection regulations but also to enable data harmonization across previously fragmented practices \citep{acconcia_pubblicazione_2023}.

Today, the GNA serves as a centralised platform for a broad community of users: public administrators and government officials, who rely on it for regulatory oversight; professional archaeologists and cultural heritage consultants, who use it for research and field documentation; and stakeholders involved in public works, including national infrastructure planners, for whom it facilitates informed decision-making within the constraints of heritage protection.

For instance, major entities like TERNA (the national electricity grid operator), RFI (the Italian railway network), or the Milan Metro rely on the platform to assess archaeological constraints before launching construction projects. The platform helps them identify archaeological sites, deposits, and or protected areas that must be preserved.

Central to the platform is a QGIS template, which standardises data entry and visualisation. This tool supports collaborative integration of local information into the national infrastructure, offering users a unified territorial overview. It enables the comparison of diverse archaeological records, improves the quality of evaluations, and promotes transparency across institutional workflows. Thanks to its open-source foundation and modular structure, the GNA continues to evolve based on user feedback, maintaining a shared national standard while accommodating diverse local contributions \citep{calandra_il_2023, boi_il_2023}.

\subsection{User Manual and Operational Support}\label{sec:gna_manual}
To guide consultants in correctly navigating the \textit{Geoportale} platform, a collaboratively maintained user manual (\textit{manuale operativo}) is made available online through a MediaWiki environment hosted on the GNA server \citep{gna_wiki_2024}. This living document offers structured instructions on all aspects of data input, visualization, and management within the GNA.

The manual offers step-by-step instructions for compiling and submitting data using the QGIS template, including the creation and editing of project modules (MOPR), the documentation of archaeological sites and events (MOSI), and the proper use of supporting layers such as risk maps or thematic overlays. Each section of the manual is designed to be accessible both to GIS beginners and to experienced professionals, offering annotated screenshots, workflow examples, and direct links to downloadable resources. A notable feature of the operational manual is its integration with the GNA QGIS plugin,\footnote{The GNA Plugin enables interaction with the platform to directly load data related to a specific Project Module (MOPR) into QGIS and to submit the Project Module back to GNA, making it quickly available to everyone. The address for the official repository is: \url{https://gna.cultura.gov.it/qgis/plugins/plugins.xml}.} which allows users to directly download standardised data layers -- such as archaeological risk assessments, site boundaries, or previous project records -- into their local GIS environment \citep{gabucci_template_2023}.

In addition to the written documentation, the GNA offers continuous operational support through a dedicated Help Desk service, coordinated by Ada Gabucci.\footnote{Ada Gabucci is a specialist in Roman-period archaeology, with expertise in stratigraphic methods, northern Italian material culture, and the structuring of archaeological data. She has over thirty years of experience consulting for public institutions, including the Italian Ministry of Culture (ICCD, ICA, DG-ABAP), its regional branches, the Veneto Region, and several universities, including Trieste, Venice, Verona, Bologna, Genova, and Pisa. Her work also encompasses cultural heritage cataloguing, ministerial regulations, and the design of complex Geographic Information Systems \parencite{noauthor_ada_2025}.} Users encountering technical challenges or seeking clarification on data entry procedures can contact the Help Desk for personalised assistance. Combined with the collaborative and evolving character of the manual, the Help Desk sustains a genuine community of practice, promoting the exchange of expertise, and nurtures the ongoing refinement of the platform’s tools and resources.

\section{Proof of Concept}
In response to the challenges users face in quickly locating relevant information when accessing and navigating the GNA operative manual, as well as the high volume of inquiries received by the Help Desk, the need emerged for a more intelligent and scalable support solution. To meet this demand, we\footnote{The project was carried out in the context of a curricular internship, which took place between December 2024 and May 2025. The work involved several actors: the author of this thesis as the intern, responsible for the design, implementation, and documentation of the system; Mario Caruso, Head of Research and Development at \href{https://www.bupsolutions.com/en/home_en/}{BUP Solutions},\nocite{https://www.bupsolutions.com/en/home_en/} who first conceived the idea for the project; Simone Persiani, AI Specialist at BUP, who provided technical guidance and support for the implementation of the RAG pipeline; and Ada Gabucci from the Ministry of Culture (MiC), responsible for the Help Desk of the \href{https://gna.cultura.gov.it}{GNA}, who contributed domain expertise and took part in the final feedback session.} developed an AI-powered information system in the form of a question-answering assistant, designed both to assist users directly and to alleviate the workload of the Help Desk. Drawing on the current state of AI, ML and DH methodologies -- as discussed in \autoref{chap:sota} and especially \autoref{sec:evol_qas} --, RAG was chosen as the most effective approach. This technology equips the GNA AI assistant to dynamically access the GNA corpus and produce answers that are precise, contextually grounded, and tailored to user needs.

\subsection{Functional Requirements}
Functional requirements specify the concrete capabilities the system must provide in order to meet the needs of its users and stakeholders, outlining the core actions through which it delivers value. These features are detailed as follows:
\begin{itemize}
    \item \textbf{Natural language understanding (NLU):} the system must interpret user queries phrased in natural language, supporting diverse question types (factoid, list, explanatory, etc.) and handling both simple and complex multipart queries.
    \item \textbf{Information retrieval:} the system must retrieve relevant passages or document segments from the GNA knowledge base, using vector similarity search over chunked content.
    \item \textbf{Answer generation:} the system must synthesise coherent, context-aware answers using RAG, drawing from retrieved passages and maintaining reference to original sources.
    \item \textbf{Source attribution and citation:} answers must include traceable citations (e.g., URLs) to ensure transparency and support verification.
    \item \textbf{Conversational memory:} the system must retain context from previous exchanges to handle follow-up questions and maintain dialogue continuity within a session.
    \item \textbf{Multilingual support:} the system must process and generate responses in Italian, with potential extensibility to other languages.
    \item \textbf{Interactive user interface:} users must be able to input queries and view answers through an accessible web interface, including features such as clickable citations, feedback buttons, and session management.
    \item \textbf{User feedback collection:} the system must provide mechanisms for users to rate responses and submit qualitative feedback, enabling ongoing evaluation and improvement.
\end{itemize}

\subsection{Non-Functional Requirements}
Non-functional requirements define how the system should operate to ensure quality, usability, and maintainability:
\begin{itemize}
    \item \textbf{Accuracy and relevance:} answers must be factually correct, directly address user queries, and reference up-to-date information.
    \item \textbf{Performance and scalability:} the system must deliver responses with low latency (target average retrieval and response time inferior to 1 second per query) and scale to support multiple concurrent users.
    \item \textbf{Robustness and reliability:} the system should gracefully handle invalid queries, errors, and resource constraints without crashing.
    \item \textbf{Transparency and traceability:} every generated answer must cite its sources clearly. The underlying process for retrieval should be auditable.
    \item \textbf{Security and privacy:} the system must securely handle sensitive data. User interactions should be anonymised, and no personally identifiable information should be stored.
    \item \textbf{Maintainability and extensibility:} The architecture must support modular updates (e.g., changing retrieval strategies), and facilitate maintenance, debugging, and future enhancements.
    \item \textbf{Resource efficiency:} the solution must operate efficiently within the limits of available hardware, minimising memory and compute consumption, especially for cloud deployment scenarios without GPU access.
    \item \textbf{User accessibility:} the web interface must be usable by non-technical users and meet accessibility standards (e.g., clear labelling, visible focus indicators, consistent navigation) \parencite{noauthor_web_nodate}.
    \item \textbf{Continuous evaluation:} the system must support automated and human-in-the-loop evaluation methodologies, generating reports on retrieval accuracy, answer quality, and user satisfaction over time.
\end{itemize}


\end{spacing}
\chapter{Methodology}
\label{chap:methodology}
\begin{spacing}{1.5}
This chapter details the methodological workflow for designing and implementing the GNA QA system. The system leverages a RAG pipeline tailored to the Geoportale Nazionale per l’Archeologia (GNA) knowledge base (KB). It comprises modular components for data acquisition, preprocessing, retrieval, generation, feedback collection, and evaluation. The methodology evolved through iterative development: beginning with a prototype built on LangChain and advancing to a full-scale system with custom components optimised for resource efficiency and multilingual support.


\section{Prototype}
The initial prototype served as a proof-of-concept integrating core RAG elements using off-the-shelf tools \citep{mishra_using_2024,akkiraju_facts_2024}. The pipeline combined:
\begin{itemize}
      \item a CSV-based knowledge base,
      \item a FAISS vector store,
      \item the Mistral NeMo large language model,
      \item and a Streamlit-based interface.
\end{itemize}

The prototype uses LangChain for managing prompts, memory, and asynchronous streaming.\footnote{LangChain is an open-source framework designed to simplify the development of applications powered by LLMs. For further details and practical examples, consult the official documentation at \url{https://python.langchain.com/docs/introduction/}.} The interface allows users to input NL queries in Italian and receive fluent, context-aware responses.\nocite{noauthor_langchain_2024}

Evaluation was conducted via a dual approach:
\begin{enumerate}
      \item \textbf{Human Assessment}, using a 5-point Likert scale to rate and annotate consistency, fluency, completeness, and relevance \citep{abeysinghe_challenges_2024};
      \item \textbf{LLM-as-a-Judge}, where GPT-3.5 is used to auto-evaluate responses via few-shot prompting.\footnote{See also \citep{svikhnushina_approximating_2023} for method inspiration.}
\end{enumerate}

Challenges included limited scalability, resource inefficiency, lack of chunk-level metadata control and the absence of standardised
evaluation methods and benchmarks. These findings informed the redesign of the full system.

\section{Full-scale implementation}
The full system was re-engineered from scratch to support dynamic, scalable document ingestion, contextual retrieval, and high-precision answer generation using open-source LLMs. All LangChain dependencies were removed in favor of custom Python implementations to improve modularity, debugging transparency, and flexibility in processing. The final architecture includes:
\begin{itemize}
      \item a custom knowledge base construction module, which includes sitemap generation and web-crawling,
      \item semantic chunking and metadata enrichment,
      \item LLM-based vector embeddings,
      \item a FAISS vector store for retrieval,
      \item a generation module with open-source Mistral NeMo model,
      \item generative response with inline citation handling,
      \item a reactive front-end interface built on Streamlit,
      \item and a feedback management system.
\end{itemize}

\sloppy
\section{Data Acquisition and Preprocessing}
\subsection{Sitemap Generation}
The sitemap is constructed via a focused breadth-first crawler targeting the MediaWiki documentation (\url{https://gna.cultura.gov.it/wiki}) of the GNA \citep{mic_mic_2019}. 
The crawler:
\begin{itemize}
      \item starts at the root node (\texttt{\href{https://web.archive.org/web/20250803092155/https://gna.cultura.gov.it/wiki/index.php/Pagina_principale}{Pagina\_principale}}\nocite{noauthor_wiki_2025});
      \item follows internal links matching \texttt{/wiki/index.php/}, excluding namespaces such as \texttt{Special:}, \texttt{User:}, or \texttt{Talk:};
      \item removes query parameters to avoid duplicates;
      \item applies a polite crawling policy (1-second delay, custom user-agent header);
      \item imposes crawl depth (max 10) and page limits (max 200 pages);
      \item and generates a structured sitemap in XML format.
\end{itemize}

HTML is parsed using BeautifulSoup, isolating the main content via focusing on the \texttt{div} with \texttt{id="mw-content-text"}, and excluding sidebars and footers. The output is serialised into an XML file (\texttt{GNA\_\_sitemap.xml}) including last-modified timestamps, priority, and change frequency. 

This sitemap serves as the foundation for subsequent document crawling and chunking.

\subsection{Document Crawling and Chunking}
Crawling retrieves the URLs listed in the sitemap. The system fetches HTML content asynchronously, applying retries and concurrency limits. During parsing:
\begin{itemize}
      \item extraneous HTML elements are stripped;
      \item headers (\texttt{h1-h6}), paragraphs, tables, and images are preserved in semantic order;
      \item a hierarchical structure is reconstructed to retain navigational breadcrumbs.
\end{itemize}
      
Content is then chunked using a sliding window strategy of max-512 characters per chunk and 128-character overlap. Each chunk includes metadata such as \texttt{source URL}, \texttt{page title}, \texttt{section headers}, \texttt{chunk ID}, \texttt{content type} (text, table, image), \texttt{keywords}, \texttt{named entities}.\footnote{Output: 835 structured chunks saved in \texttt{data/chunks\_memory.json}.}

NER is performed using spaCy (\texttt{it\_core\_news\_md}), while keywords are extracted with KeyBERT multilingual model (\textit{paraphrase-multilingual-MiniLM-L12-v2}). Tables are chunked as standalone elements, and image references are retained for OCR via Tesseract when applicable.

This contributes to creating the knowledge base, which is stored in a JSON file (\texttt{data\/knowledge\_base.json}) for subsequent embedding, retrieval and generation tasks.

\sloppy
\subsection{Vector Embeddings}
Document chunks are converted into dense vector representations using the \textit{intfloat\/multilingual-e5-large} model from Sentence Transformers. This model was selected for its multilingual encoding capabilities and strong performance in semantic retrieval tasks, making it suitable for the predominantly Italian knowledge base while allowing also cross-lingual queries. 

Text chunks are processed in batches and transformed into L2-normalised embeddings to ensure vector magnitudes are uniform. Embeddings are cached locally to avoid redundant computation across runs. The normalised vectors are stored in a FAISS \texttt{IndexFlatIP} index, which performs brute-force nearest neighbor search using the inner product (dot product) as metric. Alongside the vector index, a separate metadata store is maintained, linking each embedding to its corresponding chunk through a unique identifier. This separation enables efficient similarity search in FAISS while preserving quick access to rich metadata such as source URL, document structure, and content type for downstream processing.\footnote{Output: FAISS index stored in \texttt{.faiss\_db}, linked with its metadata.}

These embeddings and their associated metadata form the foundation for the retrieval stage, where user queries -- also encoded with the same \textit{multilingual-e5-large} model, to guarantee that both queries and documents share the same normalised vector space -- are matched against the stored vectors to identify the most semantically relevant chunks for answer generation.

\section{Candidates Retrieval}
When a user submits a query, it is embedded using the same encoder to ensure vector space consistency. The FAISS index, configured for inner-product similarity, is queried to return the top-k candidate chunks.\footnote{The default value of \texttt{k=5} was determined empirically to balance response quality and token constraints.} Retrieval is executed entirely within the vector space to maximise speed and maintain consistent scoring across CPU-based deployments. The retrieved results are enriched with their stored metadata, which includes source URL, document title from the original web section, hierarchical section headings, and content type (text, table, image). Candidates are then grouped by provenance, ensuring that related chunks from the same source URL are passed together into the generation stage, thus improving contextual coherence, supporting inline citation, and reducing redundancy.

To further improve factual density, a lightweight filtering heuristic is applied to penalise very short or contextless chunks, deprioritising fragments that lack substantive information. The grouped and filtered candidates are returned as structured context objects, ready to be consumed by the answer generation module. 

This retrieval framework serves as the baseline for subsequent ablation studies described in the evaluation phase (cf. \autoref{sec:exp_ablation}), where alternative retrieval strategies and scoring variations are tested against this reference implementation.

\subsection{Experimental Setup for Ablation Studies}\label{sec:exp_ablation}
To systematically evaluate the contribution of various retrieval strategies, an experimental setup was implemented following an ablation logic. In this context, ablation refers to the process of selectively removing or isolating individual components to measure their specific impact on overall performance. None of the evaluated approaches was integrated into the main system; instead, each was tested independently to allow for a broader performance comparison, as detailed in \autoref{sec:evaluation_protocol}. 

The following retrieval configurations were tested:
\begin{itemize}
    \item \textbf{Dense:} uses dense vector embeddings for retrieval. It wraps the FAISS vector database (via \texttt{VectorDatabaseWrapper}) and returns the top-k chunks ranked by embedding-based similarity scores. Queries are cached using a normalized MD5 hash to avoid recomputation, and batch querying is supported.
    \item \textbf{BM25:} employs the BM25 algorithm for traditional keyword-based retrieval. The index is built over concatenated fields -- \texttt{title}, \texttt{keywords}, \texttt{headers\_context}, and \texttt{document} -- from the same metadata store. Text is preprocessed for Italian (stopword removal, stemming, and clitic/apocope handling), then tokenized. Batch mode is also supported.
    \item \textbf{Hybrid:} combines the dense retriever and BM25 to balance semantic and lexical matching. We tested two fusion strategies:
    \begin{itemize}
            \item \textbf{Weighted RRF:} aggregates ranks from both retrievers using Weighted Reciprocal Rank Fusion (RRF). The fusion score for document $d$ is computed as
\[
\frac{w_\mathrm{dense}}{k + \mathrm{rank}_\mathrm{dense}} + \frac{w_\mathrm{sparse}}{k + \mathrm{rank}_\mathrm{sparse}}, \quad k = 60
\]
with default weights $w_{\text{dense}} = w_{\text{sparse}} = 1.0$, \text{\texttt{candidate\_k}} = 50, and \text{\texttt{top\_k}} = 5.
            \item \textbf{Score-blend:} merges normalized scores from both retrievers using a custom blending function that allows for fine-tuning the influence of each method:
\[
S_{\text{norm},d} = \frac{S_d - \min(S_d)}{\max(S_d) - \min(S_d)}, \quad  
S_{\text{norm},s} = \frac{S_s - \min(S_s)}{\max(S_s) - \min(S_s)}
\]
\[
S_h = S_{\text{norm},d} + \alpha \cdot S_{\text{norm},s}
\]
where $\alpha$ controls the influence of the sparse retriever ($w_d = w_s = 1$, $k = 60$). \citep{wang_searching_2024} Unlike RRF, this method requires compatible score scales between retrievers.\footnote{Edge-case behaviors include: \textbf{a)} \textit{Document appears in only one list}: in Score-blend, it keeps its normalized score (the missing side contributes zero), while in RRF, it receives only one rank term and typically ranks lower than documents appearing in both lists; \textbf{b)} \textit{All scores equal in a list}: in Score-blend, normalization produces identical values, so BM25 adds little influence, while RRF still differentiates by rank order.}
\end{itemize}
\end{itemize}
%\noindent\subsection*{\textit{\Large What to pick?}}
\vspace{1em}
\noindent{\textit{\Large What to pick?}}

The choice between Weighted RRF and Score-blend depends on the specific retrieval context. RRF is particularly suitable when score scales between retrievers are incompatible or unstable, as its rank-based aggregation is less sensitive to scale differences and prioritizes consensus across retrieval methods. Conversely, Score-blend is more appropriate when per-query scores are reliable, as it allows fine-grained control over the relative influence of dense and sparse components, enabling more tailored retrieval behaviour.\\


\noindent Additional ablation experiments included:\\

\noindent\subsubsection*{\Large Query Rewrite}
Query Rewrite is a retrieval enhancement technique designed to reformulate the user’s input query in order to increase the likelihood of retrieving relevant documents. In the context of these experiments, query rewriting is realised as a multi-strategy process that generates alternative query variants through complementary transformations, each targeting different aspects of query understanding and expansion \citep{li_dmqr-rag_2024}. Specifically, the approach integrates:
\begin{itemize}
      \item \textbf{Core Content Extraction (CCE)} -- a sequence-to-sequence transformation, using the \textit{it5-small model}, that rewrites the query to capture its essential informational content while removing peripheral terms.
      \item \textbf{Keyword Expansion (QE)} -- key terms are extracted with KeyBERT and enriched with n-gram combinations and synonym substitutions to introduce semantically related expressions.
      \item \textbf{General Query Rewriting (GQR)} -- this process is based on spaCy functionalities of lemmatization and stopword removal, producing a normalized lexical form of the query.
      \item \textbf{Pseudo-Relevance Feedback (PRF)} -- top-ranked documents from an initial retrieval pass are analysed to extract additional high-frequency terms not present in the original query, which are then appended to form an expanded query.
      \item \textbf{Query Decomposition} -- conjunctive or disjunctive queries are split into simpler sub-queries, each covering a distinct semantic aspect.
\end{itemize}

These strategies can be applied individually or in combination (\texttt{strategy="all"}), producing a set of reformulated queries. Each reformulated query is submitted to the base retriever (Dense, BM25, Hybrid variants) to have the resulting candidate documents.\\


\noindent\subsubsection*{\Large Rerank}

Rerank is a post-retrieval refinement process that reorders an initial set of candidate documents according to a more precise relevance estimation. In our implementation, this stage operates as a wrapper over a base retriever (Dense, BM25, or Hybrid) and uses a transformer-based cross-encoder model (\textit{cross-encoder/ms-marco-MiniLM-L-6-v2}) to jointly encode the query and each candidate document, producing a contextual relevance score. Unlike the base retriever, which typically evaluates query–document similarity using independent embeddings or lexical term matching, the cross-encoder considers full cross-attention between query and document tokens, enabling a richer semantic alignment.

At runtime, the reranker receives the top-N candidates from the base retriever (with N controlled by \texttt{max\_rerank\_candidates}, set to 50), tokenizes each query–document pair, and performs inference in batches with mixed-precision support when available. The raw model outputs are interpreted as relevance scores, and candidates are sorted accordingly, producing a final top-k list with improved ordering accuracy.


\section{Generation}
The generation phase employs Mistral NeMo,\footnote{\url{https://web.archive.org/web/20250803120348/https://mistral.ai/news/mistral-nemo}.\nocite{noauthor_mistral_2025}} an open-source LLM accessible via a dedicated Mistral API and hosted independently. The choice of this model was driven by multiple factors:
\begin{itemize}
      \item open-source availability and permissive license,
      \item strong performance on multilingual tasks,
      \item low latency and high throughput on CPU hardware,
      \item availability of an official API for direct deployment integration.
      \item ability to handle long context windows -- e.g., up to 128 thousands tokens, which is sufficient for processing multiple retrieved chunks.
\end{itemize}

Among available open-source LLMs evaluated -- e.g., LLaMA 3, OpenAI, Falcon --, only Mistral NeMo satisfied all criteria in terms of language coverage, response control, and reproducibility, while also providing infrastructure for model fine-tuning and evaluation in research contexts.\footnote{For an overview between open-source and proprietary solutions, see: \textcite{noauthor_open_2025}.}

The generation module is designed to produce fluent, context-aware answers with inline citations. It uses a custom prompt template that includes system instructions (see \autoref{sec:prompt_engineering}), user query, top-k retrieved chunks (grouped and cited), chat history and memory (to support context-aware follow-up).

The API request includes temperature, top-p, and max token constraints, with defaults of:
\begin{itemize}
      \item temperature = 0.3, to ensure factuality;
      \item top-p = 0.9, to control diversity;
      \item max-tokens = 512, to limit response length.
\end{itemize}

Responses are post-processed to verify the inclusion of inline citations, language alignment for Italian, and basic formatting (e.g., numbered citations, paragraph boundaries).

\subsection{Prompt Engineering Techniques} \label{sec:prompt_engineering}
The system uses structured prompt engineering to ensure accurate, traceable, and contextually coherent answers. The prompt template is dynamically generated with the following components:

\subsubsection*{System Instructions, Boundaries and Constraints}
A custom system message (\autoref{fig:system_prompt}) is injected at the top of the prompt to guide the model’s behavior. This message instructs the system to enforce neutrality in its answers, prioritise relevant and verifiable information, and include inline citations in square brackets that correspond to metadata entries. It also explicitly discourages hallucinations and speculative responses.

\begin{figure}[H]
\begin{Verbatim}[breaklines=true]
   system_content = """
        Sei un assistente virtuale incaricato di rispondere a domande sul manuale operativo del Geoportale Nazionale per l'Archeologia (GNA), gestito dall'Istituto Centrale per il Catalogo e la Documentazione (ICCD).

        Segui sempre queste regole:
        1. Non rispondere a una domanda con un'altra domanda.
        2. Rispondi **sempre** in italiano, indipendentemente dalla lingua della domanda, a meno che l'utente non richieda esplicitamente un'altra lingua.
        3. Cita le fonti utilizzando la notazione [numero] dove:
            - Le fonti sono fornite nel contesto della domanda e sono numerate in ordine crescente;
            - Usa numeri diversi per fonti diverse;
            - Non includere mai l'URL nel corpo della risposta;
        4. Alla fine della risposta, aggiungi un elenco di riferimenti con il seguente formato, su righe separate:
            [ID] URL_completo
        5. Se non hai informazioni sufficienti per rispondere, rispondi "Non ho informazioni sufficienti".

        Le tue risposte devono essere sempre:
        - Disponibili, professionali e naturali
        - Grammaticalmente corrette e coerenti
        - Espresse con frasi semplici, evitando formulazioni complesse o frammentate
        - Complete e chiare, evitando di lasciare domande senza risposta
        """
\end{Verbatim}
\caption{System prompt specifying assistant constraints and response instructions.}\label{fig:system_prompt}
\end{figure}

\subsubsection*{Dynamic Citation Handling}

Each chunk passed to the LLM is numbered and grouped with its metadata (title, URL). When generating a response, Mistral is instructed to cite only the chunks used, ensuring traceability. Post-processing checks for unmatched citations or unreferenced metadata.

This modular prompting strategy proved crucial for maintaining factual consistency while supporting multilingual input and long-form reasoning.


\section{Evaluation Protocol}\label{sec:evaluation_protocol}
Evaluation was conducted across two dimensions: 
\begin{itemize}
      \item \textbf{Quantitative}, using metrics such as Recall (R@), Mean Reciprocal Rank (MRR@), Normalized Discounted Cumulative Gain (nDCG@), Average Precision (AP@), and Latency to assess retrieval performance;
      \item \textbf{Qualitative}, gathering user feedback on response relevance, fluency, completeness, and usability.
\end{itemize}

This dual perspective follows the recognition that effective RAG-based chatbots require not only accurate retrieval and generation but also operational efficiency and adaptability \citep{akkiraju_facts_2024}. The evaluation process is designed to be iterative, allowing for continuous refinement of the system based on user interactions and performance metrics.

The evaluation faced several structural limitations:
\begin{itemize}
      \item \textbf{Absence of a golden standard:} there was no authoritative, verified set of responses to serve as an absolute accuracy benchmark.
      \item \textbf{No baseline system:} internal institutional tasks lacked comparable legacy solutions or predefined benchmarks.
      \item \textbf{Lack of real users or domain experts:} the system was initially developed without direct input from actual users, limiting the applicability of findings. Human evaluation by real end-users was integrated later in the process.
      \item \textbf{Limited applicability of traditional automated metrics:} common algorithmic measures such as BLEU, ROUGE, and METEOR are widely used in text generation but have been shown to be ineffective in dialogue and QA contexts \citep{deriu_survey_2020,liu_how_2016}.
\end{itemize}

Given these limitations, ..... aligning with recommendations from recent RAG evaluation literature.


In this phase several challenges arose due to the absence of standardised
evaluation methods and benchmarks. The lack of a golden standard – a verified set of responses representing objective truth – made it difficult to comprehensively assess accuracy. Additionally, there were no predefined benchmarks for internal institutional tasks, nor a baseline system for comparison, as real users or domain experts were unavailable for testing at this stage. While taxonomies for LLM evaluation exist (Guo et al., 2023), they are not universally applicable to chatbot and QA assessment. Moreover,
commonly used algorithmic metrics such as BLEU, ROUGE, and METEOR are
considered ineffective for dialogue systems (Deriu et al., 2019; Liu et al., 2016), reinforcing the need for human evaluation as a complementing approach (Mehri and
Eskenazi, 2020; Abeysinghe, 2024).

Akkiraju (2024) fifteen RAG pipeline control points, empirical results on accuracy-latency tradeoffs between large and small LLMs.

\subsection{Datasets}\label{sec:datasets}

Two synthetic evaluation sets were created:
\begin{itemize}
      \item \textbf{Single-hop dataset:} 508 queries designed to elicit single-document answers, each with a single gold document. This set tests the system's ability to retrieve and generate answers based on isolated chunks of information.
      \item \textbf{Combined dataset:} 400 additional queries that require multi-hop reasoning, where answers are derived from multiple documents (2-4 chunks), for a total of 908 queries entries. This set evaluates the system's capacity to integrate information from various sources and generate coherent, contextually rich responses.
\end{itemize}

Tasks:


\subsection{Metrics}\label{sec:metrics}

\citep{liu_how_2016} for not using automed metrics like BLEU, ROUGE, etc.

\begin{itemize}
    \item R@5 (Recall at 5): Measures the fraction of relevant documents retrieved in the top 5 results.
    \item MRR (Mean Reciprocal Rank): Evaluates the rank position of the first relevant document.
    \item nDCG@5 (Normalised Discounted Cumulative Gain at 5): Measures ranking quality, emphasising higher placement of relevant results.
    \item AP@5 (Average Precision at 5): Computes the average of precision values at each relevant document within the top 5.
    \item Latency: Average retrieval time per query (in seconds).
\end{itemize}

\subsection{Experimental Setup for Ablation Studies}
Describe the retrieval configurations tested (Dense, BM25, Hybrid + Weighted RRF, Hybrid + Score-blend) and how they differ.
\begin{itemize}
    \item Dense: Uses dense vector embeddings for retrieval.
    \item BM25: Employs the BM25 algorithm for traditional keyword-based retrieval.
    \item Hybrid + Weighted RRF: Combines dense and BM25 results using a weighted Reciprocal Rank Fusion (RRF) strategy.
    \item Hybrid + Score-blend: Merges scores from both methods using a custom blending function.
    \item Query Rewrite: Applies query rewriting to improve retrieval relevance.
    \item Rerank: Applies a reranking step to refine the top results based on relevance.
\end{itemize}
Ablations tested the effect of enabling/disabling query rewriting and reranking across both datasets, measuring R@5, MRR, nDCG@5, AP@5, and latency.
The design aligns with Akkiraju et al.’s emphasis on control-point evaluation, ensuring that performance changes can be tied to specific retrieval and orchestration steps rather than observed only at the output level.

Note whether query rewriting and reranking were applied in each configuration, as shown in your table.



\section{User Interface}
The user interface (UI) is implemented using Streamlit, selected for its fast prototyping capabilities, built-in support for asynchronous processing, and integration with Python-based NLP components. The front end acts as the primary touchpoint between users and the GNA QA system, displaying generated answers with inline citations, and collecting user feedback.

Why the UI matters methodologically (collecting user feedback, enabling qualitative eval).

One screenshot max;

\subsection{Streamlit User Experience}
The Streamlit app is organised into three main areas:
\begin{enumerate}
      \item \textbf{Sidebar:} Contains MiC refernce, institutional links to the GNA documentation, and contextual help describing the assistant’s capabilities. It also provides functional controls including a \texttt{Clear Chat History} button to reset the session (\texttt{st.session\_state.chat\_history}) and a \texttt{Download Feedback} button for exporting user queries and system's responses.
      \item \textbf{Main interface:} Provides a natural language input field (\texttt{st.chat\_input}) for querying the assistant. It displays the chat history, including:
      \begin{itemize}
            \item user messages,
            \item assistant responses (formatted via \texttt{st.chat\_message}),
            \item feedback buttons (3 points Likert-scale) for each assistant reply.
      \end{itemize}
      \item \textbf{Session features}: 
      \begin{itemize}
            \item Chat history is limited to the most recent 10 exchanges (\texttt{MAX\_HISTORY}), which are stored and updated in \texttt{st.session\_state};
            \item Feedback from individual message indexes is stored as a set (\texttt{st.session\_state.feedback\_given}), allowing the system to prevent duplicate ratings and dynamically update UI feedback (e.g., ``Valutazione registrata con successo.'').
      \end{itemize}
\end{enumerate}

\subsection{Session State Management}
To ensure smooth, stateful interactions and to minimise computational overhead, the system makes extensive use of \texttt{st.session\_state}. This approach allows chat memory for both user and assistant messages to persist, enables caching of API results -- including responses generated by Mistral and citation mappings -- and facilitates the tracking of feedback.\footnote{For more details on Streamlit session state management, see \url{https://docs.streamlit.io/}.\nocite{noauthor_streamlit_2025}} In addition, the system incorporates Python’s \texttt{asyncio} and \texttt{concurrent.futures} modules to support non-blocking retrieval and generation. For each interaction with the language model, a new event loop is created to maintain compatibility with Streamlit’s execution model, and the query to the model is run in a thread-safe environment using a \texttt{ThreadPoolExecutor}. This pattern guarantees UI responsiveness and provides a fluid user experience even under high network latency or API response times.

\section{Feedback Loop}
To support iterative improvement of the assistant and promote user engagement, the system integrates an interactive feedback module that allows users to rate each answer directly within the Streamlit interface. This design is intended to support continuous quality assessment and transparent evaluation of LLM-generated content.

\subsection{Collection}
Each assistant response is immediately followed by three clickable UI buttons in the form of a 3-point Likert scale, enabling users to provide a quick evaluation:
\begin{itemize}
\item 1 star $\star$ - Poor: The answer is incorrect, incomplete, or irrelevant.
\item 2 stars $\star\star$ - Fair: The answer is partially correct but lacks clarity or depth.
\item 3 stars $\star\star\star$ - Good: The answer is accurate, complete, and well-structured.
\end{itemize}

This mechanism is implemented using Streamlit’s interactive widgets. Once a rating is submitted, the system prevents duplicate feedback using an in-memory tracking set (\texttt{st.session\_state.feedback\_given}). The interface then displays a confirmation message (e.g., ``\textit{Valutazione registrata con successo.}''), improving transparency.

\subsection{Storage and Export}
Feedback is stored locally in a SQLite database (\texttt{feedback.db}) with the following schema:

\begin{figure}[H]
\begin{Verbatim}[breaklines=true]
                  CREATE TABLE feedback (
                        id INTEGER PRIMARY KEY AUTOINCREMENT,
                        timestamp TEXT NOT NULL,
                        message_index INTEGER NOT NULL,
                        question TEXT NOT NULL,
                        answer TEXT NOT NULL,
                        rating INTEGER NOT NULL
                  );
\end{Verbatim}
\caption{Schema of the feedback's SQL databse.}\label{fig:sql_schema}
\end{figure}

This structure supports reproducibility and traceability by maintaining a clear mapping between: the user's query string, the generated response, the rating score (1-3) and the associated timestamp (i.e., time of submission). All records are saved with minimal overhead, using parameterised SQL insertions and transaction-safe commits.\\

To ensure long-term preservation and collaborative accessibility of user feedback, the system implements a mechanism for periodic synchronization of collected feedback with a persistent repository. This setup enables version control over user interaction logs, supports iterative evaluation by external reviewers, and enables rollback and comparison across model updates.\footnote{This implementation is intended for controlled research use only. For production environments, secure alternatives such as authenticated APIs and hardened database infrastructures are recommended to ensure data safety and compliance with privacy standards.}\\

From the sidebar, users can export all feedback as a \texttt{.csv} file using the \textit{``Esporta feedback''} button. This functionality is powered by the \texttt{export\_feedbacks()} function, which queries the database and converts it to a downloadable format using Pandas. 

This export can be used by researchers, developers, or project coordinators to assess the assistant's performance over time.


\section{Resource and Deployment Constraints}
Deploying a multilingual RAG workflow with different language models integrated and semantic search poses non-trivial challenges for environments lacking access to GPUs or large memory allocations. To ensure the GNA QA system remains responsive and cost-efficient, especially for open usage, several optimization strategies were integrated into the pipeline.

\subsection{Memory Management}
To reduce RAM usage during embedding, retrieval, and generation, the following practices were adopted:
\begin{itemize}
\item \textbf{Garbage collection routines:} Explicit calls to Python's garbage collector (\texttt{gc.collect()}) were introduced to free unused memory between embedding and response generation steps.
\item \textbf{Batch processing:} Document chunks are processed in batches to minimise memory overhead, especially during retrieval.
\item \textbf{Lazy loading:} All embeddings are computed once and stored to disk. On app startup, only metadata is loaded, and the FAISS index is memory-mapped to reduce RAM footprint.
\item \textbf{Asynchronous processing:} Streamlit's async capabilities allow the UI to remain responsive while background tasks are executed, preventing memory spikes during long-running operations.
\item \textbf{Cache clearing policies:} \texttt{st.session\_state} objects are pruned after each session or on manual reset by the user to prevent memory bloating during prolonged use.
\end{itemize}

\subsection{Computational Constraints Mitigation}
Given the constraint of CPU-only deployments on platforms like Streamlit Cloud, the following strategies were implemented:
\begin{itemize}
\item \textbf{Model selection:} The use of \textit{intfloat/multilingual-e5-large} for embedding provides a trade-off between semantic accuracy and compute efficiency, even without GPU acceleration.
\item \textbf{API offloading:} Offloading generative tasks to an external API prevents the local system from being overloaded and allows scaling independently of frontend performance.
\item \textbf{Timeout and fallback handlers:} If the generation request exceeds 10 seconds or fails -- e.g., due to API rate limits --, the app returns a graceful fallback response, allowing users to retry or simplify their query without crashing the app.
\item \textbf{Asynchronous I/O:} For embedding, retrieval, and response generation, asynchronous requests reduce UI freezing and ensure smoother user experience even under high latency conditions.
\end{itemize}


\section{Ethics and Data Governance}
The GNA QA system is designed with a strong emphasis on ethical considerations and data governance, particularly in the context of cultural heritage and public information. Key principles include:
\begin{itemize}
\item \textbf{Transparency:} The system provides clear information about its data sources, methodologies, and limitations, ensuring users understand how answers are generated and the provenance of information.
\item \textbf{Privacy:} The system avoids processing personally identifiable information (PII) and ensures that all data used is publicly available or explicitly licensed for use in research and educational contexts.
\item \textbf{Licensing:} All components, including the knowledge base, models, and software libraries, are selected based on permissive licenses that allow for academic and non-commercial use, ensuring compliance with legal and ethical standards.
\item \textbf{Auditability:} The system maintains logs of user interactions, feedback, and system performance, enabling ongoing evaluation and improvement while respecting user privacy and data protection regulations.
\end{itemize}
These principles guide the development and deployment of the GNA QA system, ensuring it serves as a responsible and ethical tool for cultural heritage public engagement.
Provenance, PII avoidance, licensing, auditability.


\end{spacing}

\chapter{Results}
\label{chap:results}
\pdfcomment{WIP}
\begin{spacing}{1.5}


\section{Ablation Studies}\label{sec:retrieval_ablation}


\begin{table}[H]
\centering
\resizebox{\linewidth}{!}{%
\setlength{\tabcolsep}{3pt}
\footnotesize
\begin{tabular}{l c c *{10}{c}} % Method, Query rewrite, Rerank, then 10 numeric columns
\toprule
& & & \multicolumn{5}{c}{\textbf{SINGLE-HOP}} & \multicolumn{5}{c}{\textbf{COMBINED (single+multi-hop)}} \\
\cmidrule(lr){4-8}\cmidrule(lr){9-13}
\textbf{Method} & \shortstack[c]{\textbf{Query}\\\textbf{rewrite}} & \textbf{Rerank} & R@5 & MRR & nDCG@5 & AP@5 & Latency & R@5 & MRR & nDCG@5 & AP@5 & Latency \\
\midrule
Dense   &  \xmark     & \xmark & 67.51 & \underline{47.04} & 52.18 & \underline{47.04} & \underline{0.29}  & 50.78 & 48.21 & 53.70 & 34.98 & \underline{0.19} \\
        &  \checkmark & \xmark & 58.07 & 36.76 & 42.09 & 36.76 & 5.98 & 42.64 & 35.41 & 41.32 & 26.24 & 3.10   \\
        & \xmark      &  \checkmark  & 45.47 & 25.41 & 30.36 & 25.41 & 1.30  & 34.57 & 27.71 & 32.90 & 19.48 & 0.82 \\
        & \checkmark  &  \checkmark  & 52.55 & 31.66 & 36.87 & 31.66 & 4.76 & 38.16 & 30.45 & 36.0 & 22.51 & 3.46     \\
\addlinespace
BM25                          & \xmark & \xmark & 65.15 & 43.56 & 48.98 & 43.56 & \textbf{0.001} & 53.09 & \textbf{51.23} & \textbf{57.13} & 35.50 & \textbf{0.001} \\
                              & \checkmark & \xmark & 57.87 & 37.37 & 42.50 & 37.37 & 1.98 & 46.65 & 42.15 & 48.33 & 29.38 & 1.20     \\
                              & \xmark      &  \checkmark  & 45.47 & 25.41 & 30.36 & 25.41 & 1.30  & 34.57 & 27.71 & 32.90 & 19.48 & 0.82 \\
                              & \checkmark  &  \checkmark  & 43.89 & 25.87 & 30.35 & 25.87 & 10.02 & 35.18 & 30.33 & 35.68 & 20.59 & 4.96     \\
Hybrid                        &          &  &  &  &  &  &  &  &  &  &  &      \\
\hspace{0.5em}\textit{+ Weighted RRF}          & \xmark   & \xmark & \underline{69.68} & 46.72 & \underline{52.49} & 46.72 & 0.33  & \textbf{53.98} & \underline{50.93} & \underline{56.92} & \underline{36.15} & 0.32 \\
                              & \checkmark & \xmark & 57.48 & 37.48 & 42.48 & 37.48 & 4.38  & 43.52 & 38.41 & 44.10 & 27.68 & 3.21     \\
                              & \xmark      &  \checkmark  & 43.50 & 24.70 & 29.33 & 24.70 & 1.67  & 33.06 & 26.36 & 31.26 & 18.70 & 0.87 \\
                              & \checkmark  &  \checkmark  & 38.58 & 21.14 & 25.47 & 21.14 & 6.51 & 29.98 & 23.95 & 28.66 & 16.44 & 6.75     \\
\addlinespace
\hspace{0.5em}\textit{+ Score-blend}   & \xmark   & \xmark & \textbf{70.27} & \textbf{48.59} & \textbf{54.02} & \textbf{48.59} & 0.55 & \underline{53.35} & 50.84 & 56.48 & \textbf{36.69} & 0.45     \\
                              & \checkmark & \xmark & 57.67 & 37.17 & 42.31 & 37.17 & 4.57 & 43.61 & 38.16 & 43.87 & 27.52 & 3.99   \\
                              & \xmark      &  \checkmark  & 43.70 & 24.89 & 29.53 & 24.89 & 2.64 & 33.14 & 26.21 & 31.14 & 18.70 & 1.22   \\
                              & \checkmark  &  \checkmark  & 38.58 & 21.47 & 25.72 & 21.47 & 6.44 & 30.13 & 23.98 & 28.82 & 16.55 & 4.8   \\
\bottomrule
\end{tabular}%
}
\caption{Results for different retrieval methods on test datasets. Best per column is bold and the second-best is underlined. The latency is measured in seconds per query.}
\label{tab:benchmark}
\end{table}

Provide a short interpretation — e.g., which configuration excels in which scenario, trade-offs between speed and accuracy, and performance differences between single-hop and combined datasets.

\section{Qualitative analysis}
Results showed that:

65\% of answers were rated as “Relevant”,

25\% as “Partially relevant”,

10\% as “Not relevant”.

Feedback indicated that relevance dropped when:

the query used ambiguous phrasing,

or too few document chunks were retrieved due to vector sparsity.

Users appreciated:

the traceability of answers via citations,

the lightweight UI,

and the multilingual support.


\end{spacing}
\chapter{Discussion}
\label{chap:discussion}
\pdfcomment{WIP}
\begin{spacing}{1.5}

\section{Further Development}

specializzazione sul dominio archeologia

\end{spacing}
\chapter{Conclusion}
\label{chap:conclusion}
\pdfcomment{WIP}
\begin{spacing}{1.5}


\end{spacing}

\clearpage
% Appendices
\begin{appendices}
    \chapter{Implementation details}
    \label{appendix:A}

In this study, all the experiments have been performed on a system with an Intel Core i7-1185G7 CPU at 3.00 GHz, 16 GB of RAM, and integrated Intel Iris Xe Graphics with 128 MB of VRAM. Additionally, the following software packages have been used to implement the proposed approach: PyTorch, NumPy, Pandas, Matplotlib, SpaCy, HuggingFace...

\chapter{Abbreviations}
\label{appendix:B}
\autoref{tab:abbreviations} describes the abbreviations and acronyms used throughout the chapters.


\begin{table}[H]
\centering
\footnotesize
\begin{tabularx}{\textwidth}{ l l >{\justifying\arraybackslash}X }
\toprule
\textbf{Abbreviation} & \textbf{Full Form} & \textbf{Glossary Definition} \\
\midrule
AI    & Artificial Intelligence & The field of computer science dedicated to creating systems capable of performing tasks that typically require human intelligence, such as reasoning, learning, and problem-solving. \\
\cmidrule(lr){1-2}
DH    & Digital Humanities & An interdisciplinary field that applies computational methods and tools to humanities research, analysis, and dissemination. \\
\cmidrule(lr){1-2}
QA    & Question-Answering & A technology or task in which a system provides precise answers to questions posed in natural language. \\
\cmidrule(lr){1-2}
QAS   & Question-Answering System & A system designed to answer questions automatically by processing natural language input, often using methods from IR and NLP. \\
\cmidrule(lr){1-2}
RAG   & Retrieval-Augmented Generation & An approach combining information retrieval with generative models, allowing AI to reference external data sources when generating answers. \\
\cmidrule(lr){1-2}
LLM   & Large Language Model & A neural network trained on massive text corpora to generate or understand human language, such as GPT or BERT. \\
\cmidrule(lr){1-2}
IR    & Information Retrieval & The process of searching, retrieving, and ranking relevant documents or data from large collections based on user queries. \\
\cmidrule(lr){1-2}
KB    & Knowledge Base & A structured collection of information or data, often used to support reasoning, search, or retrieval in AI systems. \\
\cmidrule(lr){1-2}
NLP   & Natural Language Processing & The area of AI focused on enabling computers to understand, interpret, and generate human language. \\
\cmidrule(lr){1-2}
ML    & Machine Learning & A subset of AI that involves training algorithms to recognize patterns and make decisions based on data. \\
\cmidrule(lr){1-2}
TF-IDF & Term Frequency-Inverse Document Frequency & A statistical measure used in IR to evaluate how important a word is to a document relative to a corpus, balancing term frequency and document rarity. \\
\cmidrule(lr){1-2}
BM25  & Best Match 25 & A ranking function used in IR to estimate the relevance of documents to a given search query, based on term frequency and document length normalization. \\
\cmidrule(lr){1-2}
TREC  & Text REtrieval Conference & An ongoing series of workshops and evaluations focused on advancing research in text retrieval and related tasks. \\
\cmidrule(lr){1-2}
RNN   & Recurrent Neural Network & A type of neural network architecture designed to process sequential data by maintaining a form of memory of previous inputs. \\
\cmidrule(lr){1-2}
LSTM  & Long Short-Term Memory & A special kind of RNN capable of learning long-range dependencies, often used for tasks like language modeling or time series prediction. \\
\cmidrule(lr){1-2}
CRF   & Conditional Random Field & A probabilistic graphical model used for structured prediction, especially in NLP tasks such as sequence labeling. \\
\cmidrule(lr){1-2}
SVM   & Support Vector Machine & A supervised machine learning algorithm used for classification and regression, which finds the optimal boundary between classes in the feature space. \\
\bottomrule
\end{tabularx}
\caption{Abbreviations and acronyms used in the thesis with their full forms and definitions.}
\label{tab:abbreviations}
\end{table}


\end{appendices}


\chapter*{Acknowledgments}
\addcontentsline{toc}{chapter}{Acknowledgments}

% Back Matter
\backmatter
\pagestyle{fancy}
\cleardoublepage
\phantomsection
\addcontentsline{toc}{chapter}{Bibliography}
\printbibliography

\end{document}