\chapter{Discussion}
\label{chap:discussion}
\pdfcomment{WIP}
\begin{spacing}{1.5}

Explain what metrics values ~\% indicate for the GNA QA system.

Discuss trade-offs: precision vs. recall in heritage QA; speed vs. accuracy; LLM hallucination vs. retrieval grounding.

\section{Strenghts}
Scalable crawling and chunking pipeline.

Transparent retrieval with source citations.

Lightweight implementation (CPU-friendly, deployable on Streamlit).

\section{Weaknesses}

\section{Relatio to Previous Work}
Compare findings with studies in the bibliography; Show how this project aligns with or diverges from existing QA and RAG applications.

\section{Implications for Digital Humanities}
Value of RAG systems for cultural heritage platforms (accessibility, democratising archaeology knowledge).

Tension between automation and scholarly authority (machine-generated vs. curated responses).

Potential role as support tool for researchers, students, and the public.

\end{spacing}