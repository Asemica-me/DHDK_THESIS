\chapter{Abstract}
\label{chap:abstract}
\begin{spacing}{1.5}
At the confluence of Artificial Intelligence and Digital Humanities, this thesis explores the deployment of retrieval-augmented generation (RAG) to facilitate access to the \textit{Geoportale Nazionale Archeologia (GNA)}, the Italian central repository of archaeological data. The study presents the design, implementation, and assessment of a dedicated question-answering system which integrates semantic embeddings, hybrid retrieval mechanisms, transformer-based language models, and user feedback loops into a modular pipeline.

Evaluation combined quantitative benchmarking with qualitative analysis by expert users, yielding results that underscore both the promise and the vulnerabilities of RAG in a cultural heritage context. The system improved access to procedural guidelines and technical documentation, accompanied by a reduction in misleading or extraneous information. Nonetheless, experiments revealed sensitivity to document structure and inconsistencies in provenance tracking, together with the challenge of balancing computational efficiency against contextual fidelity.

Far from claiming semantic comprehension, the system positions itself as a mediating tool that orients archaeologists and heritage professionals within vast textual corpora, surfacing relevant passages and easing interpretive navigation. Beyond the archaeological domain, its broader significance emerges in demonstrating how technical innovation intersects with infrastructural limitations and ethical imperatives. In this light, AI appears not as a surrogate for scholarly judgment, but as a means of extending humanistic inquiry and fostering renewed modes of interpretation and engagement across the digital humanities.

\vspace{\baselineskip} % Add vertical space before keywords
\noindent\textbf{Keywords:} Digital Humanities \textperiodcentered\ Information Retrieval \textperiodcentered\ Question-Answering Systems \textperiodcentered\ Retrieval-Augmented Generation \textperiodcentered\ Machine Learning \textperiodcentered\ Natural Language Processing \textperiodcentered\ Humanistic AI \textperiodcentered\ Cultural Heritage.

\end{spacing}