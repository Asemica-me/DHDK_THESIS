\chapter{Abstract}
\label{chap:abstract}
\pdfcomment{WIP}
\begin{spacing}{1.5}
This dissertation presents the design and development of a question-answering (QA) system tailored to the Geoportale Nazionale Archeologia (GNA). The research addresses the challenge of extracting relevant information from archaeological documentation using Retrieval-augmented generation (RAG) amd natural language processing (NLP) techniques. The methodology combines transformer-based language models with domain-specific information extraction to enable intuitive, natural language querying of technical documentation related to archaeological data.

% Key findings demonstrate significant improvements in retrieval accuracy compared to traditional search methods, with a 25\% increase in F1-score. 

% conclusion

\vspace{\baselineskip} % Add vertical space before keywords
\noindent\textbf{Keywords:} Digital Humanities \textperiodcentered\ Information Retrieval \textperiodcentered\ Question-Answering Systems \textperiodcentered\ Retrieval-Augmented Generation \textperiodcentered\ Natural Language Processing \textperiodcentered\ AI \textperiodcentered\ Cultural Heritage.

\end{spacing}