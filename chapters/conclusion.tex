\chapter{Conclusion}
\label{chap:conclusion}
\pdfcomment{WIP}
\begin{spacing}{1.5}

further development and future work

Summary of Contributions.

State what you built (end-to-end QA system for GNA).

Emphasise technical and scholarly contributions (pipeline, evaluation framework, deployment).

Reconnect to the thesis questions/objectives (e.g., “Can RAG improve access to archaeological knowledge?”).

Show how these results answer them.

FUTURE DIRECTIONS.

Technical improvements: advanced retrieval (re-ranking, hybrid search), larger/more specialised LLMs, multilingual expansion.

Evaluation: collaboration with domain experts for annotated datasets, user studies.

Digital Humanities applications: integration into museum/archives portals, scholalry editions, educational tools, comparison with traditional catalogues.

specializzazione sul dominio archeologia

Final Reflections.

Situate your work as a bridge between AI techniques and DH practices.

Emphasise the potential of RAG systems to enrich cultural heritage accessibility, while noting the importance of critical human oversight.

End on a hopeful note: your project shows a feasible, scalable, and adaptable model for future DH knowledge infrastructures.


\end{spacing}