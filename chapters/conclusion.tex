\chapter{Conclusion}
\label{chap:conclusion}
\begin{spacing}{1.5}
This thesis has presented the design, implementation, and evaluation of an end-to-end question-answering system for the \textit{Geoportale Nazionale Archeologia (GNA)}, showing how retrieval-augmented generation (RAG) can be applied to improve access to highly specialised cultural heritage documentation. While developed for a specific institutional use case, the GNA AI assistant offers insights of broader relevance, both technical and scholarly, into the evolving role of AI within the digital humanities.

The contributions are threefold. First, the system itself: a functioning assistant powered by AI, that systematically integrates crawling, chunking, semantic embedding, retrieval, generation, and feedback into a coherent architecture, tested across synthetic test datasets and live user interactions. Second, the evaluation framework: a dual approach that combined intrinsic retrieval metrics with human-centred assessments, making visible both capability at system level and performance as experienced by end users. Third, the methodological reflections: an exploration of the trade-offs -- between precision and recall, fluency and completeness, speed and accuracy -- that define the practical life of RAG systems in real-world contexts.

The results confirm that RAG can enhance access to dense archaeological resources, producing responses with a swifter and more accessible character than the laborious manual consultation of numerous records of technical documentation. At the same time, they underscore the fragility of such systems. Performance is highly sensitive to corpus structure, domain vocabulary, and infrastructural constraints; remain crucial yet challenging to sustain; and evaluation practices, when based on synthetic data, risk diverging from the texture of real user needs. The GNA QA system should therefore not be mistaken for a surrogate of scholarly interpretation. Its contribution is more modest but also more practical: it acts as an intermediary agent between users and documents, disclosing relevant passages, clarifying information flows, and enabling quicker orientation within complex materials.

Looking ahead, several fronts stand out for further development:
\begin{itemize}
    \item Technical refinements will be required. In particular, retrieval recall, especially for multi-hop queries, must be improved through higher candidate thresholds or repacking strategies. Provenance mechanisms need stabilisation, so that citations become reliable and trust is reinforced. 
    \item Long-term adaptability must be addressed, with strategies for updating embeddings as the knowledge base evolves and for handling security and privacy more explicitly.
    \item Evaluation must move toward more consistent collaboration with archaeologists and cultural heritage professionals, ensuring that system performance is judged not only by technical metrics but also by disciplinary relevance.
\end{itemize}

These improvements will be essential if RAG systems are to move from prototypes into robust infrastructures within the cultural heritage sector. However, what matters most here reaches beyond the realm of technical optimisation, as the contribution of this project aims at heading further than incremental adjustments. 

This work shows how the design of AI systems, whether in cultural heritage or other domains, is shaped by the intersection of methodological choices, infrastructural realities, and ethical considerations. In this light, the integration of RAG into heritage infrastructures is not merely a technical upgrade, but a reconfiguration of how knowledge is accessed, contextualised, and mediated. Archaeology, with its dispersed and stratified documentation, offers a particularly vivid test case: here, RAG has the capacity to weave fragmented records into coherent accounts oriented around user queries. Yet such promise is inseparable from risk -- the risk of oversimplification, of introducing artificial coherence, of hallucination. For this reason, critical human oversight remains indispensable, ensuring that these tools serve as mediators rather than substitutes in the interpretation of cultural knowledge.

Future research should extend this orbit, setting new vectors toward the multitude of possible directions, each illuminating a different constellation of inquiry. One promising outlet lies in multimodal retrieval, where text, images, and geospatial data are brought together within unified pipelines. Another is the development of models adapted to specific domains, multilingual in scope and capable of reflecting the linguistic and cultural diversity of heritage data. Equally important are collaborative evaluation frameworks that integrate field's experts into the assessment process, producing annotated datasets and conducting systematic user studies that better capture real information needs.

The scope of applications, too, should broaden: from integration within GLAM realities, and digital scholarly editions, to comparative studies with traditional cataloguing practices and educational settings. Such deployments would offer a fuller test of the scalability and adaptability of RAG systems across heterogeneous cultural contexts.

At a more conceptual level, the work opens up pressing questions. Can provenance be safeguarded without undermining usability? How can retrieval-generation pipelines scale without collapsing under their own complexity? A further issue lies in evaluation: in what ways might synthetic protocols be refined to better approximate the often messy and unpredictable demands of real-world information needs? And, perhaps most crucially, how might AI systems be designed not to replace but to complement, extend and reinforce the interpretive practices that remain at the very heart of humanistic inquiry?

The GNA AI assistant does not resolve these questions, but it makes them tangible. It shows that generative AI, when grounded in contextual retrieval, can serve as a pragmatic partner in the stewardship of cultural heritage, expanding access while respecting the epistemic specificities of the domain. More broadly, it illustrates a path forward in which AI functions less as a substitute for human understanding and more as a companion -- one that structures access, enhances discoverability, and supports informed engagement with complex resources.

In this sense, the project is less an endpoint than an invitation -- a doorway toward new questions, opportunities, and directions for future research. It offers a modest yet concrete contribution to the dialogue between generative AI and the humanities, a small stone laid in what may become a much larger hall of digital scholarship, portending toward a future in which AI technologies are not peripheral tools but structural beams within scholarly and heritage settings. The challenge now is to shape this integration with care, so that the ingenuity of technical design remains attuned to the subtleties of human knowledge. That challenge, and its possibilities, are both a weight and a promise. They will chart the course of the next explorations, where the paths of machine computation and human interpretation continue to converge, diverge, and intertwine.


\end{spacing}