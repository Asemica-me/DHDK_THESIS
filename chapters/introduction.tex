\chapter{Introduction}
\label{chap:introduction}
\begin{spacing}{1.5}  % line spacing
In the swiftly evolving intersection of artificial intelligence (AI) and Digital Humanities, computational methods have profoundly transformed access to and interpretation of cultural heritage resources. Question-answering systems (QAS), driven by advances in natural language processing (NLP) and retrieval-augmented generation (RAG), have become increasingly significant tools, offering new possibilities for engaging with extensive documentation and complex cultural repositories. This thesis emerges directly from an applied research experience conducted during an internship at \href{https://www.bupsolutions.com/en/home_en/}{BUP Solutions}\nocite{bup_solutions_bup_nodate}, aimed at exploring the practical feasibility and effectiveness of these AI technologies in the context of cultural heritage. Specifically, the project focused on developing a specialized question-answering (QA) system for the Geoportale Nazionale Archeologia (GNA), Italy’s primary repository for archaeological data.

The motivation for this study initially arose from a concrete, practical challenge: enabling efficient, intuitive, and contextually accurate access to the extensive and often fragmented archaeological documentation hosted by GNA. Archaeologists, heritage managers, and scholars regularly face difficulties in navigating vast volumes of technical reports, field notes, and geospatial data. In response, this project experimented with applying cutting-edge NLP techniques -- primarily Transformer-based language models -- to dynamically retrieve and synthesize relevant information based on user queries expressed in natural language.

Central to the chosen methodology is retrieval-augmented generation (RAG), an approach that significantly enhances traditional QA systems through the dynamic retrieval of domain-specific content, which augments the generative capabilities of language models. Instead of relying solely on internal model knowledge, RAG-based systems integrate external document retrieval with generative text production, resulting in greater reliability and contextually grounded responses -- crucial qualities for scholarly and professional uses within cultural heritage. While this approach inherently promises increased accuracy and reduced hallucinations compared to purely generative methods, it also involves several practical complexities and uncertainties, which were encountered firsthand during the development and evaluation phases.

Rather than adopting a narrowly theoretical or idealized perspective, this study reflects the exploratory and evolving nature of the projetc's hands-on experience. The experimental implementation followed an iterative process, shaped by trial-and-error, heuristic adjustments, and pragmatic solutions dictated by real-world constraints -- such as computational limitations, the lack of standardized evaluation benchmarks, and the linguistic complexity of the domain. The experience highlighted the tension between the ambition of AI-driven solutions and the realities of their application in complex cultural domains. In systems like the GNA’s AI assistant, the focus necessarily shifts from abstract notions of “understanding” to measurable outcomes. Success is not defined by whether the system truly comprehends archaeology, but by whether it can retrieve relevant information efficiently, navigate domain-specific nuances, and support users in making informed decisions.

The project also confronts deeper epistemological questions about what it means for an AI system to "perform" in a knowledge-intensive field. Drew McDermott’s cautionary perspective \citep{mcdermott_artificial_1976} -- highlighting the frequent gap between AI researchers' ambitious terminologies and their systems’ actual capabilities -- resonates particularly with this experience. Throughout the project, the challenges of translating theoretical expectations into functional implementations became increasingly evident. These difficulties underscored both the potential and the limitations of current AI technologies, particularly when applied to the depth and subtlety of cultural domains. Viewed through this lens, the project reveals not only what such systems can accomplish, but also what they are not yet equipped to achieve. 

In light of this reality, this study deliberately avoids overstating the system’s semantic or interpretive capabilities. Instead, it foregrounds the project’s experimental and exploratory nature, acknowledging both methodological achievements and encountered limitations. The resulting QAS represents a pragmatic yet innovative step toward applying AI in the digital humanities, offering insights into the real-world challenges and possibilities of using retrieval-augmented generation in cultural heritage contexts.

Ultimately, this project remains fundamentally hopeful. It demonstrates that even with inherent methodological challenges and necessary improvisations, AI-driven tools such as RAG-based QAS hold substantial promise for enhancing access to cultural heritage information. Through a transparent presentation of both the strengths and shortcomings discovered during this internship experiment, this thesis aims to contribute realistically yet optimistically to the ongoing dialogue between artificial intelligence and humanistic inquiry, offering a balanced vision of AI’s evolving role in supporting cultural heritage scholarship.

\end{spacing}