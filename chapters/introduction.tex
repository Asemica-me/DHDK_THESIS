\chapter{Introduction}
\label{chap:introduction}
\begin{spacing}{1.5}  % line spacing
At the swiftly evolving intersection of Artificial Intelligence (AI) and Digital Humanities (DH), computational methods have profoundly transformed access to and interpretation of cultural heritage resources. Question-answering systems (QAS), driven by advances in natural language processing (NLP) and retrieval-augmented generation (RAG), have become increasingly significant tools, offering new possibilities for engaging with extensive documentation and complex cultural repositories. This thesis emerges directly from an applied research experience conducted during an internship at \href{https://www.bupsolutions.com/en/home_en/}{BUP Solutions}\nocite{bup_solutions_bup_nodate}, aimed at exploring the realistic feasibility and effectiveness of these AI technologies in the context of cultural heritage. Specifically, the project focused on developing a specialized question-answering (QA) system for the Geoportale Nazionale Archeologia (GNA), Italy’s primary repository for archaeological data.

The motivation for this study initially arose from a concrete, practical challenge: enabling efficient, intuitive, and contextually accurate access to the extensive and often fragmented archaeological documentation hosted by GNA. Archaeologists, heritage managers, and scholars regularly face difficulties in navigating vast volumes of technical reports, field notes, operational procedures, and geospatial data. In response, this project experimented with applying cutting-edge NLP and machine learning (ML) techniques -- primarily Transformer-based language models and advanced information retrieval methods -- to dynamically retrieve and synthesize relevant information based on user queries expressed in natural language.

Central to the chosen methodology is retrieval-augmented generation (RAG), an approach that significantly enhances traditional QA systems through the dynamic retrieval of domain-specific content, which augments the generative capabilities of language models. Instead of relying solely on internal model knowledge, RAG-based systems integrate external document retrieval with generative text production, resulting in greater reliability and contextually grounded responses -- crucial qualities for scholarly and professional uses. While this approach inherently promises increased accuracy and reduced hallucinations compared to purely generative methods, it also involves several complexities and uncertainties, which were encountered firsthand during the development and evaluation phases, as will be discussed in the following chapters.

Rather than adopting a narrowly theoretical or idealized perspective, this study reflects the exploratory and evolving nature of hands-on experimentation, shaped by iterative cycles of trial-and-error, heuristic adjustments, and pragmatic responses to practical constraints such as computational limits, the absence of standardized evaluation benchmarks, and the linguistic complexity of the domain. This process brought to light the persistent tension between the ambitions of AI-driven solutions and the realities of applying them in intricate cultural contexts. In systems like the GNA’s AI assistant, the focus necessarily shifts from abstract notions of “understanding” to measurable outcomes: the true test is not whether the system comprehends archaeology in any human sense, but whether it efficiently retrieves relevant information, navigates domain-specific nuances, and supports users in making informed decisions. In describing system capabilities, I am mindful of McDermott’s famous warning against \textit{wishful mnemonics} in AI, reminding us to resist the temptation to label what our systems do with grand terms like “understand” and instead to critically assess and communicate the functional scope and limits of their actual achievement \citep{mcdermott_artificial_1976}.

In light of this reality, this study deliberately avoids overstating the system’s semantic or interpretive capabilities. Instead, it foregrounds the project’s exploratory nature, acknowledging both methodological achievements and encountered limitations. The outcome represents a pragmatic yet innovative step toward applying AI in the Digital Humanities, offering insights into the real-world challenges and possibilities of using retrieval-augmented generation in cultural heritage contexts.

Ultimately, this project remains fundamentally hopeful. It demonstrates that even with inherent methodological challenges, AI-driven tools such as RAG-based QAS hold substantial promise for enhancing access to cultural heritage information. Through a transparent presentation of both the strengths and shortcomings discovered during this internship experiment, this thesis aims to contribute realistically yet optimistically to the ongoing dialogue between artificial intelligence and humanistic inquiry, offering a balanced vision of AI’s evolving role in supporting cultural heritage scholarship.

\end{spacing}