\chapter{Introduction}
\label{chap:introduction}
\begin{spacing}{1.5}  % line spacing
At the swiftly evolving intersection of artificial intelligence (AI) and digital humanities (DH), computational methods have profoundly transformed access to and interpretation of cultural heritage resources. Among these, question-answering systems (QASs) -- driven by advances in natural language processing (NLP) and retrieval-augmented generation (RAG) -- have become increasingly significant tools, offering new possibilities of engaging with extensive documentation and complex repositories. This thesis arises directly from an applied research experience conducted during an internship at \href{https://www.bupsolutions.com/en/home_en/}{BUP Solutions}\nocite{bup_solutions_bup_nodate}, aimed at exploring the realistic feasibility and effectiveness of AI technologies in the context of cultural heritage. Specifically, the project focused on the desing, implementation and evaluation of a specialised QAS for the \textit{Geoportale Nazionale Archeologia (GNA)}, Italy’s primary repository of archaeological data under the auspices of ministerial authorities. For clarity, throughout this work the implemented system will be referred to interchangeably as the ``GNA QA system'' or the ``GNA AI assistant''.

The motivation of the present study stemmed from a practical challenge: facilitating efficient, intuitive, and accurate access to the extensive and often fragmented body of archaeological documentation hosted by the GNA. Archaeologists, heritage professionals and scholars working with this resource frequently face difficulties in navigating vast volumes of intricate technical reports, field notes, procedural guidelines, and complex geospatial data. In response, the project experimented with applying cutting-edge NLP and machine learning (ML) techniques -- primarily transformer-based language models combined with advanced retrieval methods -- to dynamically locate and synthesise relevant information based on user queries expressed in natural language.

Central to the chosen methodology is RAG, an approach that significantly enhances traditional QASs through the dynamic retrieval of domain-specific content, which augments the generative capabilities of language models. Instead of relying solely on internal model knowledge, systems grounded in RAG integrate external document retrieval with generative text production, resulting in greater reliability and outputs tethered in evidentiary contextual material -- crucial qualities for scholarly and professional uses. While this approach inherently promises increased accuracy and reduced hallucinations compared to purely generative methods, it also involves several complexities and uncertainties, which were encountered firsthand during the development and evaluation phases, as will be discussed in the following chapters.

Rather than adopting a narrowly theoretical or idealised perspective, this study reflects the exploratory and evolving nature of hands-on experimentation, shaped by iterative cycles of trial-and-error, heuristic adjustments, and pragmatic resolutions to practical constraints such as computational limits, the absence of standardised evaluation benchmarks, and the structural complexity of the domain. This process brought to light the persistent tension between the ambitions of AI solutions and the realities of applying them in intricate cultural contexts. In systems like the GNA’s AI assistant, the focus necessarily shifts from abstract notions of understanding to measurable outcomes: the true test is not whether the system comprehends archaeology in any human sense, but whether it efficiently retrieves relevant information, handles the complexities of the domain, and supports users in making informed decisions. Against such backdrop, one might ask: how far can technical ingenuity propel us before we run up against the unique subtleties of human knowledge and practice? Here, McDermott’s essay \textit{Artificial Intelligence Meets Natural Stupidity} offers a timely reminder, warning against the lure of \textit{wishful mnemonics} in AI and urging us to resist this inflationary language and the temptation to label what our systems do with grand terms like ``understand''. Instead, McDermott advocates for a clear-eyed assessment and communication of what these systems actually accomplish -- and an equally frank acknowledgment of where their true limits lie. Only through such intellectual honesty can the field avoid self-delusion and maintain its credibility \citep{mcdermott_artificial_1976}.

In light of this reality, this study deliberately avoids overstating the system’s semantic or interpretive capabilities. Instead, it foregrounds the project’s exploratory nature, acknowledging both methodological achievements and encountered limitations. The outcome represents a pragmatic effort toward applying AI in the digital humanities, offering insights into the real-world challenges and possibilities of using retrieval-augmented generation in cultural heritage contexts.

This work remains, at its heart, fundamentally hopeful. It demonstrates that even in the face of inherent methodological challenges, AI-driven tools carry genuine promise for enhancing access to cultural heritage resources. By presenting both the achievements and the limitations encountered along the way with transparency, this thesis seeks to contribute to the ongoing dialogue between AI and the humanities, offering a vision of AI’s evolving role as a catalyst for new forms of stewardship, interpretation, and engagement with cultural heritage.

\end{spacing}