\chapter{Abbreviations and Glossary}
\label{appendix:B}

\sloppy

\renewcommand\tabularxcolumn[1]{>{\noindent\justifying\arraybackslash}m{#1}} % justified + no indent in X columns

\footnotesize
\begin{tabularx}{\textwidth}{
  >{\raggedright\arraybackslash}p{2.5cm}
  >{\raggedright\arraybackslash}p{4cm}
  >{\noindent\justifying\arraybackslash}X
}
\caption{Abbreviations and acronyms with their full forms and definitions used in this thesis.}
\label{tab:abbreviations} \\
\\
\toprule
\textbf{Term} & \textbf{Full form} & \textbf{Glossary definition} \\
\midrule
\endfirsthead

\toprule
\textbf{Term} & \textbf{Full form} & \textbf{Glossary definition} \\
\midrule
\endhead

\midrule
\multicolumn{3}{r}{\textcolor{gray}{\emph{Continued on next page}}} \\
\endfoot

\bottomrule
\endlastfoot

AI    & Artificial intelligence & The field of computer science dedicated to creating systems capable of performing tasks that typically require human intelligence, such as reasoning, learning, and problem-solving. \\
\cmidrule(lr){1-3}
DH    & Digital humanities & An interdisciplinary field that applies computational methods and tools to humanities research, analysis, and dissemination. \\
\cmidrule(lr){1-3}
IR    & Information retrieval & The field of computer science that focuses on finding relevant information in large collections of data, typically unstructured text (like documents, web pages, or articles). \\
\cmidrule(lr){1-3}
NLP   & Natural language processing & The area of AI focused on enabling computers to understand, interpret, and generate human language. \\
\cmidrule(lr){1-3}
NLG   & Natural language generation & The process of automatically generating human-like text from structured data or models, often used in chatbots and content creation. \\
\cmidrule(lr){1-3}
NL & Natural language & The everyday language used by humans for communication, which NLP systems aim to understand and generate. \\
\cmidrule(lr){1-3}
QA    & Question answering & A task in NLP and IR that focuses on building systems capable of automatically answering questions posed in natural language. \\
\cmidrule(lr){1-3}
QAS   & Question-answering system & A system designed to answer questions automatically by processing natural language input, often using methods from IR and NLP. \\
\cmidrule(lr){1-3}
RAG   & Retrieval-augmented generation & An approach combining information retrieval with generative models, allowing AI to reference external data sources when generating answers. \\
\cmidrule(lr){1-3}
LLM   & Large language model & A type of neural network trained on massive text corpora to understand and generate human language. \\
\cmidrule(lr){1-3}
API  & Application programming interface & A set of protocols and tools that allow different software applications to communicate and interact with each other. \\
\cmidrule(lr){1-3}
GNA   & Geoportale Nazionale Archeologia & Italy's institutional repository for archaeological data, hosting extensive documentation and resources related to the country's cultural heritage. \\
\cmidrule(lr){1-3}
MiC & Ministero della Cultura & The Italian Ministry of Culture, responsible for the preservation and promotion of Italy's cultural heritage. \\
\cmidrule(lr){1-3}
MiBACT & Ministero dei Beni e delle Attività Culturali e del Turismo & The former name of the Italian Ministry of Culture, which was responsible for cultural heritage and tourism before its reorganization in 2021. \\
\cmidrule(lr){1-3}
CNR & Consiglio Nazionale delle Ricerche & The Italian National Research Council, a major public research institution that conducts scientific research across various disciplines, including cultural heritage. \\
\cmidrule(lr){1-3}
DG-Ant & Direzione Generale Archeologia, Belle Arti e Paesaggio & The Directorate General for Archaeology, Fine Arts, and Landscape within the Italian Ministry of Culture, overseeing archaeological heritage and cultural sites. \\
\cmidrule(lr){1-3}
ICA & Istituto Centrale per l’Archeologia & The Central Institute for Archaeology in Italy, established in 2016 as part of the Ministry of Culture, responsible for archaeological research and documentation. \\
\cmidrule(lr){1-3}
ICCD & Istituto Centrale per il Catalogo e la Documentazione & The Central Institute for Cataloging and Documentation, part of the Italian Ministry of Culture, responsible for cataloging cultural heritage assets and proposing best practices. \\
\cmidrule(lr){1-3}
SiGECweb &  Sistema Informativo Generale del Catalogo & A web platform that handles every stage of cultural heritage cataloguing, from standard creation and code assignment to cataloguing diverse assets and publishing records online for public access. \\
\cmidrule(lr){1-3}
GIS & Geographic information system & A computer system, including software and hardware, designed to capture, store, manipulate, analyse, manage, and present spatial or geographic data, often used in archaeology for mapping and spatial analysis. \\
\cmidrule(lr){1-3}
QGIS & Quantum GIS & A particular GIS software that is free and open-source. \\
\cmidrule(lr){1-3}
GLAM   & Galleries, Libraries, Archives and Museums & A collective term for institutions that preserve and provide access to cultural heritage in the public interest. \\
\cmidrule(lr){1-3}
KB    & Knowledge base & A structured collection of information or data, often used to support reasoning, search, or retrieval in AI systems. \\
\cmidrule(lr){1-3}
ML    & Machine learning & A subset of AI that involves training algorithms to recognise patterns and make decisions based on data. \\
\cmidrule(lr){1-3}
NER   & Named entity recognition & A subtask of NLP that identifies and classifies named entities (e.g., people, organizations, locations) in text. \\
\cmidrule(lr){1-3}
EL    & Entity linking & The process of connecting named entities in text to their corresponding entries in a knowledge base, enhancing understanding and retrieval. \\
\cmidrule(lr){1-3}
TF-IDF & Term Frequency-Inverse Document Frequency & A statistical measure used in IR to evaluate how important a word is to a document relative to a corpus, balancing term frequency and document rarity. \\
\cmidrule(lr){1-3}
BM25  & Best match 25 & A ranking function used in IR to estimate the relevance of documents to a given search query, based on term frequency and document length normalization.\\
\cmidrule(lr){1-3}
PRF   & Precision-Recall-F1 & Metrics used to evaluate the performance of classification models, where precision measures the accuracy of positive predictions, recall measures the ability to find all relevant instances, and F1 is the harmonic mean of precision and recall. \\
\cmidrule(lr){1-3}
RDF   & Resource Description Framework & A standard model for data interchange on the web, allowing structured representation of information about resources in a machine-readable format. \\
\cmidrule(lr){1-3}
SQL   & Structured Query Language & A standard programming language used for managing and manipulating relational databases, allowing users to query, insert, update, and delete data. \\
\cmidrule(lr){1-3}
SPARQL & SPARQL Protocol and RDF Query Language & A query language and protocol used to retrieve and manipulate data stored in RDF format, commonly used for querying knowledge graphs. \\
\cmidrule(lr){1-3}  
Ontology &  & A formal representation of a set of concepts within a domain and the relationships between those concepts, used to enable knowledge extraction, sharing and reuse. \\
\cmidrule(lr){1-3}
JSON  & JavaScript Object Notation & A lightweight data interchange format that is easy for humans to read and write, and easy for machines to parse and generate, often used for data exchange in web applications. \\
\cmidrule(lr){1-3}
CSV & Comma-Separated Values & A text file format used to store tabular data (numbers and text) where each row represents a record, and each column (field) is separated by a comma. \\
\cmidrule(lr){1-3}
TREC  & Text REtrieval Conference & An ongoing series of workshops and evaluations focused on advancing research in text retrieval and related tasks. \\
\cmidrule(lr){1-3}
LMIR & Language model information retrieval & A method of using language models to improve the effectiveness of information retrieval systems by leveraging their understanding of language and context. \\
\cmidrule(lr){1-3}
RNN   & Recurrent Neural Network & A type of neural network architecture designed to process sequential data by maintaining a form of memory of previous inputs. \\
\cmidrule(lr){1-3}
LSTM  & Long Short-Term Memory & A special kind of RNN capable of learning long-range dependencies, often used for tasks like language modeling or time series prediction. \\
\cmidrule(lr){1-3}
CRF   & Conditional Random Field & A probabilistic graphical model used for structured prediction, especially in NLP tasks such as sequence labelling. \\
\cmidrule(lr){1-3}
SVM   & Support Vector Machine & A supervised machine learning algorithm used for classification and regression, which finds the optimal boundary between classes in the feature space. \\
\cmidrule(lr){1-3}
Word2Vec & Word to Vector & A technique for representing words as vectors in a continuous vector space, capturing semantic relationships between words based on their context in large text corpora. \\
\cmidrule(lr){1-3}
GloVe  & Global Vectors for Word Representation & An unsupervised learning algorithm for obtaining vector representations of words, which captures global statistical information from a corpus. \\
\cmidrule(lr){1-3}
BERT  & Bidirectional Encoder Representations from Transformers & A pre-trained language model that uses the Transformer architecture to understand the context of words in a sentence by considering both left and right contexts simultaneously. \\
\cmidrule(lr){1-3}
OpenAI & & An artificial intelligence research and deployment company based in San Francisco (USA). \\
\cmidrule(lr){1-3}
GPT   & Generative Pre-Trained Transformer & A family of LLMs developed by OpenAI.\\
\cmidrule(lr){1-3}
ChatGPT & Generative Pre-trained Transformer & An application of the GPT architecture developed by OpenAI, fine-tuned for conversational interaction and instruction following, and released to the public in November 2022. \\
\cmidrule(lr){1-3}
T5    & Text-to-Text Transfer Transformer & T5 is a series of LLMs developed by Google AI and introduced in 2019. \\
\cmidrule(lr){1-3}
KeyBERT & & A keyword extraction technique using BERT embeddings to generate the keywords and keyphrases most similar to a document. \\
\cmidrule(lr){1-3}
BAAI & Beijing Academy of Artificial Intelligence & A Chinese research institute that develops and releases cutting-edge AI models. BAAI is the organization behind BGE, and also known for other large-scale AI projects. \\
\cmidrule(lr){1-3}
BGE   & BAAI General Embedding & A family of embedding models designed for dense retrieval and semantic search. \\
\cmidrule(lr){1-3}
Intfloat & Intelligent Floating Point & A research group and organization that develops open-source AI models for NLP. \\
\cmidrule(lr){1-3}
E5    & Embedding from Explicitly-Explained Supervision & A family of text embedding models developed by the research group Intfloat. \\
\cmidrule(lr){1-3}
Intfloat/e5 & & A family of models available on Hugging Face and based on the implementation of \textit{E5}. \\
\cmidrule(lr){1-3}
LLM-embedder &   & A model designed to generate embeddings for text using large language models, enhancing the quality of semantic representations for retrieval tasks. \\
\cmidrule(lr){1-3}
Embeddings &    & Dense vector representations of text that capture semantic meaning, used in various NLP tasks including retrieval and classification. \\
\cmidrule(lr){1-3}
Chunking &    & The process of breaking down text into smaller, manageable pieces or ``chunks'' to facilitate processing and analysis in NLP tasks. \\
\cmidrule(lr){1-3}
Vector database &    & A specialised database designed to store and retrieve high-dimensional vectors efficiently, often used in RAG systems for managing embeddings. \\
\cmidrule(lr){1-3}
Retriever &    & A component of a system responsible for searching and retrieving relevant documents or information from a database or corpus based on user queries. \\
\cmidrule(lr){1-3}
Ranking function &    & A mathematical function used to score and order documents based on their relevance to a given query, often employed in IR systems. \\
\cmidrule(lr){1-3}
XML & eXtensible Markup Language & A markup language used to encode documents in a format that is both human-readable and machine-readable, often used for data interchange. \\
\cmidrule(lr){1-3}
TEI & Text Encoding Initiative & A set of guidelines for encoding literary and linguistic texts in XML, providing a standardised way to represent complex textual structures. \\
\cmidrule(lr){1-3} 
MARC/RDA & Machine-Readable Cataloging / Resource Description and Access & Standards for encoding bibliographic information in a machine-readable format, widely used in libraries and information systems. \\
\cmidrule(lr){1-3}
GROBID & GeneRation Of BIbliographic Data & A machine learning library for extracting and structuring bibliographic information from scholarly documents, often used in academic publishing and research. \\
\cmidrule(lr){1-3}
Milvus & Milvus Vector Database & An open-source vector database designed for efficient storage, indexing, and retrieval of high-dimensional vectors, commonly used in RAG systems. \\
\cmidrule(lr){1-3}
Faiss  & Facebook AI Similarity Search & A library for efficient similarity search and clustering of
dense vectors, widely used in RAG systems for indexing and searching large datasets. \\
\cmidrule(lr){1-3}
Qdrant & Qdrant Vector Database & An open-source vector database that provides efficient storage and retrieval of high-dimensional vectors, supporting hybrid search capabilities. \\
\cmidrule(lr){1-3}
DLM reranking & Deep language model reranking & Deep language model-based reranking uses fine-tuned models that jointly encode query-document pairs and classify their relevance as ``true'' or ``false''. At inference, documents are ranked by the probability of being labeled ``true''. \\
\cmidrule(lr){1-3}
HyDE  & Hypothetical Document Embeddings & A method that generates a brief, plausible answer to the query first, then embeds that ``hypothetical doc'' for retrieval. This richer proxy query improves vector search recall/precision in RAG context, especially for vague or underspecified queries. \\
\cmidrule(lr){1-3}
Hybrid Search &  & A search approach that combines vector-based retrieval with traditional keyword search, allowing for more comprehensive and context-aware results in RAG systems. \\
\cmidrule(lr){1-3}
TILDE &  & A framework designed to facilitate the development and deployment of RAG systems, providing tools for data preparation, indexing, and retrieval. \\
\cmidrule(lr){1-3}
TILDEv2 &  & An updated version of the TILDE framework, incorporating improvements in efficiency and performance. \\
\cmidrule(lr){1-3}
LTR   & Learning-to-Rank & A machine learning approach used to optimise the ranking of search results based on user interactions and relevance feedback, improving the quality of retrieved documents in RAG systems. \\
\cmidrule(lr){1-3}
Self-RAG & Self-Retrieval-Augmented Generation & A variant of RAG where the system retrieves relevant information from its own generated content, enhancing the context and accuracy of responses. \\
\cmidrule(lr){1-3}
RAGAS & Retrieval-Augmented Generation Assessment System & An open-source evaluation framework for RAG systems. It provides metrics that assess both the retrieval and generation stages, focusing on aspects such as context relevance, answer faithfulness to retrieved documents, and overall response quality. Unlike traditional text similarity metrics (e.g., BLEU, ROUGE), RAGAS is designed to capture factual accuracy and contextual appropriateness, making it better suited for evaluating RAG-based applications like question answering and conversational agents. \\
\cmidrule(lr){1-3}
AHE & Adaptive histogram equalization & A computer image processing technique designed to enhance contrast in pictures. Unlike standard histogram equalization, the adaptive approach divides the image into multiple regions, generates a separate histogram for each, and then redistributes the lightness values based on these localized histograms. \\
\cmidrule(lr){1-3}
CLAHE & Contrast limited AHE & A variant of adaptive histogram equalization in which the contrast amplification is limited, so as to reduce the problem of noise amplification. \\
\cmidrule(lr){1-3}
XAI & Explainable Artificial Intelligence & A field of AI focused on rendering the decision-making processes of AI systems transparent and understandable to humans, often used to build trust and accountability in AI applications. \\
\cmidrule(lr){1-3}
RAG-chain & Retrieval-Augmented Generation Chain & A method that links multiple RAG components in a sequence. \\
\cmidrule(lr){1-3}
ArCo & Italian Cultural Heritage Knowledge Graph & A knowledge graph representing Italian cultural heritage, providing structured information about historical sites, artifacts, and related entities. \\
\bottomrule
\end{tabularx}
