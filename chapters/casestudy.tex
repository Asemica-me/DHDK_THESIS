\chapter{Case Study: Question-Answering System for GNA}
\label{chap:casestudy}
\sloppy
\begin{spacing}{1.5} 

\section{Geoportale Nazionale per l’Archeologia}
The Geoportale Nazionale per l'Archeologia (GNA) \citep{mic_mic_2019} serves as the central online hub for collecting and sharing data resulting from archaeological investigations carried out across Italy. The project's primary goal is the creation of a dynamic archaeological map of the national territory, which is easily updatable over time, openly accessible, and designed for reuse and integration across multiple institutional and disciplinary contexts.

The inception of the GNA traces back to a 2014 \textit{Memorandum of Understanding} signed by the Ministero dei Beni e delle Attività Culturali e del Turismo (MiBACT) -- specifically the Segretariato Generale, the Direzione Generale per le Antichità (DG-Ant), and the Consiglio Nazionale delle Ricerche (CNR). This agreement laid the groundwork for a national geoportal aimed at safeguarding and enhancing cultural heritage through integrated digital infrastructure. However, it was the establishment of the Istituto Centrale per l’Archeologia (ICA) in 2016 that provided the structural and institutional foundation for the GNA. The ICA’s mandate to define standards and promote digital archaeological databases gave renewed potential to the initiative, which culminated in the launch and formal presentation of the GNA at a ministerial venue in 2019 \citep{calandra_il_2023}.

\subsection{Purpose and Scope}
As the official repository for all research activities in archaeology and preventive archaeology -- particularly those related to public infrastructure projects -- the GNA platform was established to provide a unified national access point to essential archaeological data gathered across the country. This includes the interventions listed in \autoref{tab:gna_data_sources}, all conducted under the scientific supervision of the Italian Ministry of Culture (MiC).

\begin{table}[H]
\centering
\footnotesize
\begin{tabularx}{\textwidth}{ l >{\justifying\noindent\arraybackslash}p{0.65\textwidth} }
\toprule
\textbf{Archaeological interventions} & \textbf{Description} \\
\midrule
Preventive archaeology reports & Data from excavations and surveys carried out ahead of construction projects (e.g., highways, railways, pipelines), often submitted by private firms or cultural heritage consultants. \\
\cmidrule(lr){1-2}
Assisted scientific excavations records & Results from academic digs by universities or research institutions, including documentation of stratigraphy, finds, and site interpretation. \\
\cmidrule(lr){1-2}
Accidental discoveries & Locations of fortuitous archaeological finds, such as during agricultural work or construction, reported to local heritage authorities. Typically include preliminary spatial data and descriptive reports. \\
\cmidrule(lr){1-2}
Scheduled excavations & Long-term planned investigations, often at known heritage sites, including geospatial boundaries, uncovered structures, and findings. \\
\cmidrule(lr){1-2}
Archaeological surveys & Surface survey data with GPS-tracked locations of finds, artifact scatters, and site features. \\
\cmidrule(lr){1-2}
Cultural heritage GIS layers & External datasets from institutions (regional superintendencies, local governments, ICCD), e.g., maps of protected zones, risk maps, or site inventories. \\
\cmidrule(lr){1-2}
Legacy data and digitized archives & Georeferenced digitizations of paper maps, notebooks, and archival records previously stored in non-digital formats, essential for integrating historical with current data. \\
\cmidrule(lr){1-2}
Depository locations & Georeferenced storage locations of archaeological finds (museums, storerooms) associated with sites or interventions. \\
\cmidrule(lr){1-2}
Remote sensing and aerial surveys & Drone imagery, LiDAR scans, or satellite data used to identify and map archaeological features not visible at ground level. \\
\cmidrule(lr){1-2}
Paleontological sites & A specific level dedicated to paleontological sites is currently under study for future inclusion, aiming to protect this fragile heritage. \\
\bottomrule
\end{tabularx}
\vspace{0.5em}
\caption{Types of archaeological data sources integrated into the GNA.}
\label{tab:gna_data_sources}
\end{table}

\noindent These sources, once georeferenced and structured, are integrated into the GNA using standardized metadata and visualization protocols, to allow users to view, search, and analyze information in a spatially accurate and coherent manner \citep{boi_il_2023, acconcia_pubblicazione_2023}.

\subsection{Stakeholders and Intended Users}
The development of the GNA saw significant acceleration during the COVID-19 pandemic, which provided both the urgency and institutional impetus toward the creation of a unified digital platform for managing archaeological data nationwide. This initiative built upon years of prior collaboration between key stakeholders, including the Istituto Centrale per l’Archeologia (ICA) and the Istituto Centrale per il Catalogo e la Documentazione (ICCD), who had already developed a cataloging structure to document archaeological assessments and identified sites within the Sistema Informativo Generale del Catalogo (SiGECweb) \citep{calandra_il_2023, boi_il_2023}. The pandemic underscored the limitations of purely textual cataloging and catalyzed a shift toward a more dynamic and geospatially grounded approach, leading to the adoption of a GIS-based framework better suited for preventive archaeology and territorial planning. The result was a consolidated national infrastructure designed not only to support compliance with cultural heritage protection regulations but also to enable data harmonization across previously fragmented practices \citep{acconcia_pubblicazione_2023}.

The GNA is primarily intended for use by:
\begin{itemize}
    \item Public administrators and government officials
    \item Professional archaeologists and cultural heritage consultants
    \item Stakeholders involved in public works, such as national infrastructure planners
\end{itemize}

\noindent For instance, major entities like TERNA (the national electricity grid operator), RFI (the Italian railway network), or the Milan Metro rely on the platform to assess archaeological constraints before launching construction projects. The platform helps them identify archaeological sites, deposits, and zones to avoid, ensuring the preservation of cultural heritage during the planning and development of public infrastructure. 

Central to the system is a QGIS\footnote{QGIS is a free, open-source Geographic Information System (GIS) software used for creating, managing, and analyzing geospatial data.} template that standardizes data entry and visualization. This tool supports efficient integration of local information into the national infrastructure, offering users a unified territorial overview. It enables the comparison of diverse archaeological records, improves the quality of evaluations, and promotes transparency across institutional workflows. Thanks to its open-source foundation and modular structure, the GNA continues to evolve based on user feedback, maintaining a shared national standard while accommodating diverse local contributions \citep{calandra_il_2023, boi_il_2023}.

\subsection{User Manual and Operational Support}
To guide users in correctly navigating the system, a collaboratively authored user manual (\textit{manuale operativo}) is made available through a MediaWiki environment hosted on the GNA server \citep{gna_wiki_2024}. This living document offers structured instructions for data input and visualization within QGIS, including detailed documentation for using the GNA template plugin. These tools enable users to download and integrate standardized data layers -- such as archaeological risk maps, identified sites, or project modules -- directly into their GIS workflows. Complementing the manual, a Help Desk service is managed by Ada Gabucci,\footnote{Ada Gabucci is a specialist in Roman-period archaeology, with expertise in stratigraphic methods, northern Italian material culture, and the structuring of archaeological data. She has over thirty years of experience consulting for public institutions, including the Italian Ministry of Culture (ICCD, ICA, DG-ABAP), its regional branches, the Veneto Region, and several universities, including Trieste, Venice, Verona, Bologna, Genova, and Pisa. Her work also encompasses cultural heritage cataloguing, ministerial regulations, and the design of complex Geographic Information Systems.\\See: \textcolor{teal}{\citefield{noauthor_ada_2025}{url}}.\nocite{noauthor_ada_2025}} who provides direct assistance to users facing technical difficulties or requiring clarification.

\section{Proof of Concept}
In response to the challenges users face when accessing and navigating the GNA operative manual, as well as the high volume of inquiries received by the Help Desk, a need emerged for a smarter and more efficient support solution. To address this, we developed an information system in the form of a question-answering chatbot designed to assist users directly and reduce the Help Desk’s workload. Based on the current state of AI, machine learning, and digital humanities methodologies -- as discussed in \autoref{chap:sota} and \autoref{sec:evol_qas} -- retrieval-augmented generation (RAG) combined with natural language processing (NLP) was chosen as the most effective approach. This technology enables the chatbot to dynamically retrieve relevant information, which serves as an augmented knowledge base, allowing it to generate precise, context-aware, and up-to-date answers tailored to user queries.
\\

The following chapter details the methodological framework and practical steps undertaken during the development of this chatbot system. It provides an in-depth explanation of the design choices, technical architecture, data preparation, implementation and evaluation processes.


\end{spacing}