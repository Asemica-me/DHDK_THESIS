\chapter{Case Study: Question-Answering System for GNA}
\label{chap:casestudy}
\sloppy
\begin{spacing}{1.5} 

\section{Geoportale Nazionale per l’Archeologia (GNA)}
Geoportale Nazionale per l'Archeologia (GNA) \citep{mic_mic_2019} serves as the central online hub for the collection, management, and dissemination of data generated by archaeological investigations carried out across Italy \citep{acconcia_pubblicazione_2023}. Developed under the auspices of the Ministry of Culture (MiC), the project's primary goal is the creation of a dynamic archaeological map of the national territory, which is easily updatable over time, openly accessible, and designed for reuse and integration across multiple institutional and disciplinary contexts \citep{falcone_dematerializzazione_2023}.

The inception of the GNA traces back to a 2014 \textit{Memorandum of Understanding} signed by the Ministero dei Beni e delle Attività Culturali e del Turismo (MiBACT) -- specifically the Segretariato Generale, the Direzione Generale per le Antichità (DG-Ant), and the Consiglio Nazionale delle Ricerche (CNR). This agreement laid the groundwork for a national platform dedicated to the safeguarding and enhancement of cultural heritage through integrated digital infrastructure. However, it was the establishment of the Istituto Centrale per l’Archeologia (ICA) in 2016 that provided the structural and institutional foundation for the GNA. The ICA’s mandate to define standards and promote digital archaeological databases gave renewed potential to the initiative, which culminated in the launch and formal presentation of the GNA at a ministerial venue in 2019 \citep{calandra_il_2023}.

Beyond being a data aggregator, the GNA serves as a dynamic knowledge base, collecting digital content from professional archaeologists -- especially those engaged in preventive archaeology --, research groups, universities, and concession-holders. The platform also accommodates a variety of outputs, from QGIS-based vector data to reports, documentation packages, and datasets from academic and research contexts. Data publication in the GNA is managed with attention to quality standards, intellectual property, and open-access principles, supported by the assignment of DOIs and the use of Creative Commons licensing (CC-BY 4.0), ensuring both traceability and reusability \citep{acconcia_pubblicazione_2023,falcone_dematerializzazione_2023,boi_il_2023}.

\subsection{Purpose and Scope}
As the official repository for all research activities in archaeology -- particularly those related to public infrastructure projects -- the GNA platform was established to provide a unified national access point to essential archaeological data gathered nationwide. This includes the interventions listed in \autoref{tab:gna_data_sources}, all conducted under the scientific supervision of the Italian Ministry of Culture (MiC) \citep{acconcia_pubblicazione_2023,falcone_dematerializzazione_2023}.

\addtocounter{table}{-1}
\begin{table}[H]
\centering
\footnotesize
\begin{tabularx}{\textwidth}{ l >{\justifying\noindent\arraybackslash}p{0.65\textwidth} }
\toprule
\textbf{Archaeological interventions} & \textbf{Description} \\
\midrule
Preventive archaeology reports & Data from excavations and surveys carried out ahead of construction projects (e.g., highways, railways, pipelines), often submitted by private firms or cultural heritage consultants. \\
\cmidrule(lr){1-2}
Assisted scientific excavations records & Results from academic digs by universities or research institutions, including documentation of stratigraphy, finds, and site interpretation. \\
\cmidrule(lr){1-2}
Accidental discoveries & Locations of fortuitous archaeological finds, such as during agricultural work or construction, reported to local heritage authorities. Typically include preliminary spatial data and descriptive reports. \\
\cmidrule(lr){1-2}
Scheduled excavations & Long-term planned investigations, often at known heritage sites, including geospatial boundaries, uncovered structures, and findings. \\
\cmidrule(lr){1-2}
Archaeological surveys & Surface survey data with GPS-tracked locations of finds, artifact scatters, and site features. \\
\cmidrule(lr){1-2}
Cultural heritage GIS layers & External datasets from institutions (regional superintendencies, local governments, ICCD), e.g., maps of protected zones, risk maps, or site inventories. \\
\cmidrule(lr){1-2}
Legacy data and digitised archives & Georeferenced digitizations of paper maps, notebooks, and archival records previously stored in non-digital formats, essential for integrating historical with current data. \\
\cmidrule(lr){1-2}
Depository locations & Georeferenced storage locations of archaeological finds (museums, storerooms) associated with sites or interventions. \\
\cmidrule(lr){1-2}
Remote sensing and aerial surveys & Drone imagery, LiDAR scans, or satellite data used to identify and map archaeological features not visible at ground level. \\
\cmidrule(lr){1-2}
Paleontological sites & A specific level dedicated to paleontological sites is currently under study for future inclusion, aiming to protect this fragile heritage. \\
\bottomrule
\end{tabularx}
\vspace{0.5em}
\caption{Types of archaeological data sources integrated into the GNA.}
\label{tab:gna_data_sources}
\end{table}

\noindent These sources, once georeferenced and structured, are integrated into the GNA using standardised metadata and visualization protocols, to allow users to view, search, and analyze information in a spatially accurate and coherent manner \citep{boi_il_2023, acconcia_pubblicazione_2023}.

\subsection{Stakeholders and Intended Users}
The development of the GNA saw significant acceleration during the COVID-19 pandemic, which provided both the urgency and institutional impetus toward the creation of a unified digital platform for managing archaeological data nationwide. This initiative built upon years of prior collaboration between key stakeholders, including the Istituto Centrale per l’Archeologia (ICA) and the Istituto Centrale per il Catalogo e la Documentazione (ICCD), who had already developed a cataloging structure to document archaeological assessments and identified sites within the Sistema Informativo Generale del Catalogo (SiGECweb) \citep{calandra_il_2023, boi_il_2023}. The pandemic underscored the limitations of purely textual cataloging and catalyzed a shift toward a more dynamic and geospatially grounded approach, leading to the adoption of a GIS-based framework better suited for preventive archaeology and territorial planning. The result was a consolidated national infrastructure designed not only to support compliance with cultural heritage protection regulations but also to enable data harmonization across previously fragmented practices \citep{acconcia_pubblicazione_2023}.

The GNA is primarily intended for use by:
\begin{itemize}
    \item Public administrators and government officials
    \item Professional archaeologists and cultural heritage consultants
    \item Stakeholders involved in public works, such as national infrastructure planners
\end{itemize}

\noindent For instance, major entities like TERNA (the national electricity grid operator), RFI (the Italian railway network), or the Milan Metro rely on the platform to assess archaeological constraints before launching construction projects. The platform helps them identify archaeological sites, deposits, and or protected areas that must be preserved. The GNA also supports compliance with European and Italian open data and transparency regulations, guaranteeing both civic access and the protection of intellectual property, as per national FOIA and EU directives\footnote{The FOIA (Freedom of Information Act) Guidelines are documents issued by the Italian National Anti-Corruption Authority (ANAC) to clarify and guide the implementation of the right to generalised civic access in Italy. The guidelines -- especially those from 2016 -- define the limits and exclusions to access, as well as specify the publication and transparency obligations for public administrations.\\See more at \url{https://foia.gov.it/normativa}.\nocite{noauthor_normativa_2016}} \citep{falcone_dematerializzazione_2023}.

Central to the system is a QGIS\footnote{QGIS is a free, open-source Geographic Information System (GIS) software used for creating, managing, and analyzing geospatial data.} template that standardises data entry and visualization. This tool supports efficient integration of local information into the national infrastructure, offering users a unified territorial overview. It enables the comparison of diverse archaeological records, improves the quality of evaluations, and promotes transparency across institutional workflows. Thanks to its open-source foundation and modular structure, the GNA continues to evolve based on user feedback, maintaining a shared national standard while accommodating diverse local contributions \citep{calandra_il_2023, boi_il_2023}.

\subsection{User Manual and Operational Support}
To guide users in correctly navigating the system, a collaboratively maintained user manual (\textit{manuale operativo}) is made available online through a MediaWiki environment hosted on the GNA server \citep{gna_wiki_2024}. This living document offers structured instructions on all aspects of data input, visualization, and management within the GNA platform.

The manual offers step-by-step instructions for compiling and submitting data using the QGIS template, including the creation and editing of project modules (MOPR), the documentation of archaeological sites and events (MOSI), and the proper use of supporting layers such as risk maps or thematic overlays. Each section of the manual is designed to be accessible both to GIS beginners and to experienced professionals, offering annotated screenshots, workflow examples, and direct links to downloadable resources. A notable feature of the operational manual is its integration with the GNA QGIS plugin, which allows users to directly download standardised data layers -- such as archaeological risk assessments, site boundaries, or previous project records -- into their local GIS environment \citep{gabucci_template_2023}.

In addition to the written documentation, the GNA provides ongoing operational support through a dedicated Help Desk service, coordinated by Ada Gabucci.\footnote{Ada Gabucci is a specialist in Roman-period archaeology, with expertise in stratigraphic methods, northern Italian material culture, and the structuring of archaeological data. She has over thirty years of experience consulting for public institutions, including the Italian Ministry of Culture (ICCD, ICA, DG-ABAP), its regional branches, the Veneto Region, and several universities, including Trieste, Venice, Verona, Bologna, Genova, and Pisa. Her work also encompasses cultural heritage cataloguing, ministerial regulations, and the design of complex Geographic Information Systems.\\See: \url{https://web.archive.org/web/20250724081422/https://conf24.garr.it/it/speaker/ada-gabucci}.\nocite{noauthor_ada_2025}} Users encountering technical challenges or seeking clarification on data entry procedures can contact the Help Desk for personalised assistance. This direct support, together with the collaborative and evolving nature of the manual, fosters a strong community of practice, encouraging the sharing of expertise and continual improvement of the platform’s tools and resources.

\section{Proof of Concept}
In response to the challenges users face in quickly locating relevant information when accessing and navigating the GNA operative manual, as well as the high volume of inquiries received by the Help Desk, a need emerged for a smarter and more efficient support solution. To address this, we developed an information system in the form of a question-answering system designed to assist users directly and reduce the Help Desk’s workload. Based on the current state of AI, ML and DH methodologies -- as discussed in \autoref{chap:sota} and especially \autoref{sec:evol_qas} -- RAG combined with NLP was chosen as the most effective approach. This technology enables the chatbot to dynamically retrieve relevant information, which serves as an augmented knowledge base, allowing it to generate precise, context-aware, and up-to-date answers tailored to user queries.

\subsection{Functional Requirements}
Functional requirements define what the system must do to deliver value to users and stakeholders:
\begin{itemize}
    \item \textbf{Natural language understanding (NLU):} The system must interpret user queries phrased in natural language, supporting diverse question types (factoid, list, explanatory, etc.) and handling both simple and complex multi-part queries.
    \item \textbf{Information retrieval:} The system must retrieve relevant passages or document segments from the GNA knowledge base, using vector similarity search over chunked content.
    \item \textbf{Context-aware answer generation:} The system must synthesise coherent, context-aware answers using RAG, drawing from retrieved passages and maintaining reference to original sources.
    \item \textbf{Source attribution and citation:} Answers must include traceable citations (e.g., URLs) to ensure transparency and support verification.
    \item \textbf{Conversational memory:} The system must retain context from previous exchanges to handle follow-up questions and maintain dialogue continuity within a session.
    \item \textbf{Multilingual support:} The chatbot must process and generate responses in Italian, with potential extensibility to other languages.
    \item \textbf{User feedback collection:} The system must provide mechanisms for users to rate responses and submit qualitative feedback, enabling ongoing evaluation and improvement.
    \item \textbf{Interactive user interface:} Users must be able to input queries and view answers through an accessible web interface, including features such as clickable citations, feedback buttons, and session management.
\end{itemize}

\subsection{Non-Functional Requirements}
Non-functional requirements define how the system should operate to ensure quality, usability, and maintainability:
\begin{itemize}
    \item \textbf{Accuracy and relevance:} Answers must be factually correct, directly address user queries, and reference up-to-date information.
    \item \textbf{Performance and scalability:} The system must deliver responses with low latency (target average retrieval and response time inferior to 1 second per query) and scale to support multiple concurrent users.
    \item \textbf{Robustness and reliability:} The system should gracefully handle invalid queries, errors, and resource constraints without crashing.
    \item \textbf{Transparency and traceability:} Every generated answer must cite its sources clearly. The underlying process for retrieval should be auditable.
    \item \textbf{Security and privacy:} The system must securely handle sensitive data. User interactions should be anonymised, and no personally identifiable information should be stored.
    \item \textbf{Maintainability and extensibility:} The architecture must support modular updates (e.g., changing retrieval strategies), and facilitate maintenance, debugging, and future enhancements.
    \item \textbf{Resource efficiency:} The solution must operate efficiently within the limits of available hardware, minimising memory and compute consumption, especially for cloud deployment scenarios without GPU access.
    \item \textbf{User accessibility:} The web interface must be usable by non-technical users and meet accessibility standards (e.g., clear labeling, visual feedback, keyboard navigation).
    \item \textbf{Continuous evaluation:} The system must support automated and human-in-the-loop evaluation methodologies, generating reports on retrieval accuracy, answer quality, and user satisfaction over time.
\end{itemize}

\citep{abu_shawar_chatbots_2007,arslan_survey_2024,gupta_comprehensive_2024}\\


The following chapter details the methodological framework and practical steps undertaken during the development of the system, providng in-depth explanation of the design choices, technical architecture, data preparation, implementation and evaluation processes.


\end{spacing}