\chapter{Case Study: A Question-Answering System for GNA}
\label{chap:casestudy}
\sloppy
\begin{spacing}{1.5} 

\section{Geoportale Nazionale per l’Archeologia (GNA)}
Geoportale Nazionale per l'Archeologia (GNA) \citep{mic_mic_2019} serves as the central online hub for the collection, management, and dissemination of data generated by archaeological investigations carried out across Italy \citep{acconcia_pubblicazione_2023}. Developed under the auspices of the Ministry of Culture (MiC), the project's primary goal is the creation of a dynamic archaeological map of the national territory, which is easily updatable over time, openly accessible, and designed for reuse and integration across multiple institutional and disciplinary contexts \citep{falcone_dematerializzazione_2023}.

The inception of the GNA traces back to a 2014 \textit{Memorandum of Understanding} signed by the Ministero dei Beni e delle Attività Culturali e del Turismo (MiBACT) -- specifically the Segretariato Generale, the Direzione Generale per le Antichità (DG-Ant), and the Consiglio Nazionale delle Ricerche (CNR). This agreement laid the groundwork for a national platform dedicated to the safeguarding and enhancement of cultural heritage through integrated digital infrastructure. However, it was the establishment of the Istituto Centrale per l’Archeologia (ICA) in 2016 that provided the structural and institutional foundation for the GNA. The ICA’s mandate to define standards and promote digital archaeological databases gave renewed potential to the initiative, which culminated in the launch and formal presentation of the GNA at a ministerial venue in 2019 \citep{calandra_il_2023}.

Far from being a mere data aggregator, the GNA serves as a dynamic knowledge base, collecting digital contributions from professional archaeologists -- especially those active in preventive archaeology -- as well as from research groups, universities, and concession-holders. Its scope encompasses a wide spectrum of outputs, ranging from QGIS-based vector data to reports, documentation packages, and datasets from academic and research projects. Data publication within the GNA is managed with attention to quality standards, intellectual property rights, and open-access principles, supported by the assignment of DOIs and distribution under Creative Commons licensing (CC-BY 4.0), ensuring both traceability and reusability \citep{acconcia_pubblicazione_2023,falcone_dematerializzazione_2023,boi_il_2023}. The platform is also aligned with European and Italian open data and transparency regulations, fulfilling requirements of national FOIA provisions and EU directives\footnote{The FOIA (Freedom of Information Act) Guidelines are documents issued by the Italian National Anti-Corruption Authority (ANAC) to clarify and guide the implementation of the right to generalised civic access in Italy. The guidelines -- especially those from 2016 -- define the limits and exclusions to access, as well as specify the publication and transparency obligations for public administrations.\\Read more at \url{https://foia.gov.it/normativa}.\nocite{noauthor_normativa_2016}} \citep{falcone_dematerializzazione_2023}.

\subsection{Purpose and Scope}
As the official repository for all research activities in archaeology -- particularly those related to public infrastructure projects -- the GNA platform was established to provide a unified national access point to essential archaeological data gathered nationwide. This includes the interventions listed in \autoref{tab:gna_data_sources}, all conducted under the scientific supervision of the Italian Ministry of Culture (MiC) \citep{acconcia_pubblicazione_2023,falcone_dematerializzazione_2023}.

\addtocounter{table}{-1}
\begin{table}[H]
\centering
\footnotesize
\begin{tabularx}{\textwidth}{ l >{\justifying\noindent\arraybackslash}p{0.65\textwidth} }
\toprule
\textbf{Archaeological interventions} & \textbf{Description} \\
\midrule
Preventive archaeology reports & Data from excavations and surveys carried out ahead of construction projects (e.g., highways, railways, pipelines), often submitted by private firms or cultural heritage consultants. \\
\cmidrule(lr){1-2}
Assisted scientific excavations records & Results from academic digs by universities or research institutions, including documentation of stratigraphy, finds, and site interpretation. \\
\cmidrule(lr){1-2}
Accidental discoveries & Locations of fortuitous archaeological finds, such as during agricultural work or construction, reported to local heritage authorities. Typically include preliminary spatial data and descriptive reports. \\
\cmidrule(lr){1-2}
Scheduled excavations & Long-term planned investigations, often at known heritage sites, including geospatial boundaries, uncovered structures, and findings. \\
\cmidrule(lr){1-2}
Archaeological surveys & Surface survey data with GPS-tracked locations of finds, artifact scatters, and site features. \\
\cmidrule(lr){1-2}
Cultural heritage GIS layers & External datasets from institutions (regional superintendencies, local governments, ICCD), e.g., maps of protected zones, risk maps, or site inventories. \\
\cmidrule(lr){1-2}
Legacy data and digitised archives & Georeferenced digitizations of paper maps, notebooks, and archival records previously stored in non-digital formats, essential for integrating historical with current data. \\
\cmidrule(lr){1-2}
Depository locations & Georeferenced storage locations of archaeological finds (museums, storerooms) associated with sites or interventions. \\
\cmidrule(lr){1-2}
Remote sensing and aerial surveys & Drone imagery, LiDAR scans, or satellite data used to identify and map archaeological features not visible at ground level. \\
\cmidrule(lr){1-2}
Paleontological sites & A specific level dedicated to paleontological sites is currently under study for future inclusion, aiming to protect this fragile heritage. \\
\bottomrule
\end{tabularx}
\vspace{0.5em}
\caption{Types of archaeological data sources integrated into Geoportale Nazionale per l'Archeologia.}
\label{tab:gna_data_sources}
\end{table}

\noindent These sources, once georeferenced and structured, are integrated into the GNA using standardised metadata and visualisation protocols, to allow users to view, search, and analyse information in a spatially accurate and coherent manner \citep{boi_il_2023, acconcia_pubblicazione_2023}.

What makes this material especially challenging is its variety and format. Equally significant are the operative guidelines and technical instructions that regulate how this information is produced, structured, and uploaded into the platform. For example, the MOPR (Modulo di Progetto) section provides step-by-step guidance for consultants preparing preventive archaeology reports, from the structuring of stratigraphic descriptions to the encoding of metadata fields (cf. \href{https://gna.cultura.gov.it/wiki/index.php/Compilare_il_MOPR}{Compilare il MOPR}). Similarly, technical notes (cf. \href{https://gna.cultura.gov.it/wiki/index.php/Brevi_note_su_QGIS}{Brevi note su QGIS}) illustrate the proper use of GIS software in generating and validating shapefiles before submission, underscoring the centrality of geospatial data in contemporary archaeological practice.

\subsection{Stakeholders and Intended Users}\label{sec:gna_plugin}
The development of the GNA saw significant acceleration during the COVID-19 pandemic, which provided both the urgency and institutional impetus toward the creation of a unified digital platform for managing archaeological data nationwide. This initiative built upon years of prior collaboration between key stakeholders, including the Istituto Centrale per l’Archeologia (ICA) and the Istituto Centrale per il Catalogo e la Documentazione (ICCD), who had already developed a cataloging structure to document archaeological assessments and identified sites within the Sistema Informativo Generale del Catalogo (SiGECweb) \citep{calandra_il_2023, boi_il_2023}. The pandemic underscored the limitations of purely textual cataloguing and sparked a shift toward a more dynamic and geospatially grounded approach, leading to the adoption of a GIS-based framework better suited for preventive archaeology and territorial planning. The result was a consolidated national infrastructure designed not only to support compliance with cultural heritage protection regulations but also to enable data harmonization across previously fragmented practices \citep{acconcia_pubblicazione_2023}.

Today, the GNA serves as a centralised platform for a broad community of users: public administrators and government officials, who rely on it for regulatory oversight; professional archaeologists and cultural heritage consultants, who use it for research and field documentation; and stakeholders involved in public works, including national infrastructure planners, for whom it facilitates informed decision-making within the constraints of heritage protection.

For instance, major entities like TERNA (the national electricity grid operator), RFI (the Italian railway network), or the Milan Metro rely on the platform to assess archaeological constraints before launching construction projects. The platform helps them identify archaeological sites, deposits, and or protected areas that must be preserved.

Central to the platform is a QGIS\footnote{QGIS is a free, open-source Geographic Information System (GIS) software used for creating, managing, and analysing geospatial data.} template, which standardises data entry and visualisation. This tool supports collaborative integration of local information into the national infrastructure, offering users a unified territorial overview. It enables the comparison of diverse archaeological records, improves the quality of evaluations, and promotes transparency across institutional workflows. Thanks to its open-source foundation and modular structure, the GNA continues to evolve based on user feedback, maintaining a shared national standard while accommodating diverse local contributions \citep{calandra_il_2023, boi_il_2023}.

\subsection{User Manual and Operational Support}
To guide consultants in correctly navigating the system, a collaboratively maintained user manual (\textit{manuale operativo}) is made available online through a MediaWiki environment hosted on the GNA server \citep{gna_wiki_2024}. This living document offers structured instructions on all aspects of data input, visualization, and management within the GNA platform.

The manual offers step-by-step instructions for compiling and submitting data using the QGIS template, including the creation and editing of project modules (MOPR), the documentation of archaeological sites and events (MOSI), and the proper use of supporting layers such as risk maps or thematic overlays. Each section of the manual is designed to be accessible both to GIS beginners and to experienced professionals, offering annotated screenshots, workflow examples, and direct links to downloadable resources. A notable feature of the operational manual is its integration with the GNA QGIS plugin,\footnote{The GNA Plugin enables interaction with the platform to directly load data related to a specific Project Module (MOPR) into QGIS and to submit the Project Module back to GNA, making it quickly available to everyone. The address for the official repository is: \url{https://gna.cultura.gov.it/qgis/plugins/plugins.xml}.} which allows users to directly download standardised data layers -- such as archaeological risk assessments, site boundaries, or previous project records -- into their local GIS environment \citep{gabucci_template_2023}.

In addition to the written documentation, the GNA offers continuous operational support through a dedicated Help Desk service, coordinated by Ada Gabucci.\footnote{Ada Gabucci is a specialist in Roman-period archaeology, with expertise in stratigraphic methods, northern Italian material culture, and the structuring of archaeological data. She has over thirty years of experience consulting for public institutions, including the Italian Ministry of Culture (ICCD, ICA, DG-ABAP), its regional branches, the Veneto Region, and several universities, including Trieste, Venice, Verona, Bologna, Genova, and Pisa. Her work also encompasses cultural heritage cataloguing, ministerial regulations, and the design of complex Geographic Information Systems.\\Source: \url{https://web.archive.org/web/20250724081422/https://conf24.garr.it/it/speaker/ada-gabucci}.\nocite{noauthor_ada_2025}} Users encountering technical challenges or seeking clarification on data entry procedures can contact the Help Desk for personalised assistance. Combined with the collaborative and evolving character of the manual, the Help Desk sustains a genuine community of practice, promoting the exchange of expertise, and nurtures the ongoing refinement of the platform’s tools and resources.

\section{Proof of Concept}
In response to the challenges users face in quickly locating relevant information when accessing and navigating the GNA operative manual, as well as the high volume of inquiries received by the Help Desk, the need emerged for a more intelligent and scalable support solution. To meet this demand, we developed an AI-powered information system in the form of a question-answering assistant, designed both to assist users directly and to alleviate the workload of the Help Desk. Drawing on the current state of AI, ML and DH methodologies -- as discussed in \autoref{chap:sota} and especially \autoref{sec:evol_qas} --, RAG was chosen as the most effective approach. This technology equips the GNA AI assistant to dynamically access the GNA corpus and produce answers that are precise, contextually grounded, and tailored to user needs.

\subsection{Functional Requirements}
Functional requirements specify the concrete capabilities the system must provide in order to meet the needs of its users and stakeholders, outlining the core actions through which it delivers value. These features are detailed as follows:
\begin{itemize}
    \item \textbf{Natural language understanding (NLU):} the system must interpret user queries phrased in natural language, supporting diverse question types (factoid, list, explanatory, etc.) and handling both simple and complex multipart queries.
    \item \textbf{Information retrieval:} the system must retrieve relevant passages or document segments from the GNA knowledge base, using vector similarity search over chunked content.
    \item \textbf{Answer generation:} the system must synthesise coherent, context-aware answers using RAG, drawing from retrieved passages and maintaining reference to original sources.
    \item \textbf{Source attribution and citation:} answers must include traceable citations (e.g., URLs) to ensure transparency and support verification.
    \item \textbf{Conversational memory:} the system must retain context from previous exchanges to handle follow-up questions and maintain dialogue continuity within a session.
    \item \textbf{Multilingual support:} the chatbot must process and generate responses in Italian, with potential extensibility to other languages.
    \item \textbf{User feedback collection:} the system must provide mechanisms for users to rate responses and submit qualitative feedback, enabling ongoing evaluation and improvement.
    \item \textbf{Interactive user interface:} users must be able to input queries and view answers through an accessible web interface, including features such as clickable citations, feedback buttons, and session management.
\end{itemize}

\subsection{Non-Functional Requirements}
Non-functional requirements define how the system should operate to ensure quality, usability, and maintainability:
\begin{itemize}
    \item \textbf{Accuracy and relevance:} answers must be factually correct, directly address user queries, and reference up-to-date information.
    \item \textbf{Performance and scalability:} the system must deliver responses with low latency (target average retrieval and response time inferior to 1 second per query) and scale to support multiple concurrent users.
    \item \textbf{Robustness and reliability:} the system should gracefully handle invalid queries, errors, and resource constraints without crashing.
    \item \textbf{Transparency and traceability:} every generated answer must cite its sources clearly. The underlying process for retrieval should be auditable.
    \item \textbf{Security and privacy:} the system must securely handle sensitive data. User interactions should be anonymised, and no personally identifiable information should be stored.
    \item \textbf{Maintainability and extensibility:} The architecture must support modular updates (e.g., changing retrieval strategies), and facilitate maintenance, debugging, and future enhancements.
    \item \textbf{Resource efficiency:} the solution must operate efficiently within the limits of available hardware, minimising memory and compute consumption, especially for cloud deployment scenarios without GPU access.
    \item \textbf{User accessibility:} the web interface must be usable by non-technical users and meet accessibility standards (e.g., clear labelling, visual feedback, keyboard navigation).
    \item \textbf{Continuous evaluation:} the system must support automated and human-in-the-loop evaluation methodologies, generating reports on retrieval accuracy, answer quality, and user satisfaction over time.
\end{itemize}


\end{spacing}