\chapter{Acknowledgments}

\begin{spacing}{1.5}
As I reach the end of this master’s path, I find myself filled with a mixture of emotions. There is gratitude for all I have learned, admiration for those who have inspired me, joy for the friendships I have made, and, at the same time, some (\textit{read:} a lot of!) fear of what comes next. These two years have been transformative, and I am deeply thankful for the experiences that have broadened my perspective.

My genuine appreciation is reserved to the professors and researchers whose work and teaching were inspirational, for their knowledge has encouraged me to think critically and grow with curiosity. In particular, I would like to express my gratitude to my thesis supervisor, Prof. Giovanni Colavizza, whose advice throughout this work was pivotal in shaping its outcome, and to my co-supervisor, Prof. Paolo Bonora, for his generous guidance. I am equally grateful to Mario Caruso and Simone Persiani from BUP Solutions, whose supervision during my internship was invaluable. I thank them for their precious mentorship, collaboration and patience -- from all the calls and technical discussions to the debugging sessions I had the opportunity to take part in --, for sparking the very beginning of this project and making it possible, and above all for enriching my learning experience. My thanks also go to Ada Gabucci from the GNA for her kind availability during the evaluation stage of the final system -- which by now feels less like a chatbot and more like a creature I’ve raised.

I am deeply grateful to my family for their constant encouragement and steadfast love. To \textit{mamma} Serenella, and to Gabriele: thank you for being my safe harbour, a place of care and steadiness. To my father Fausto, whose loving memory always remains as a lighthouse along my way, I owe a source of inspiration that silently sustains my path. I wish to extend my heartfelt gratitude to the other relatives, spread across the Pianura Padana and Toscana, whose support has accompanied me along the way too.

I would like to create some space as well to say thank you to my peers and to all the wonderful people I have met along the way at DHDK. The friendships formed across countries and continents are among the truest gifts of this journey, and I cherish them with affection. To the friends from the P\&R House, the command center of our countless projects and study sessions, and our harvest ground when we simply needed to have fun together and forget for a while -- thank you for making these moments brighter. Thanks are due also to all the many collaborators in the projects we developed together, for the energy, creativity, and fellowship you brought to every step.

Of course, to Maicol, for being such a true and dedicated friend of the rarest kind, your faithfulness is a gift beyond measure. My renewed thanks to both old and new friends, near and far, who have shared in laughter, motivation, and companionship throughout this venture. Each of you has enriched these years in ways I will always carry with me.

Finally, I want to acknowledge the quiet but powerful lessons that come with facing uncertainty. Stepping into programming, coding, and even revisiting maths after years immersed in Art History was anything but easy; it must be said, trading brushstrokes and artists biographies for algorithms and tensors felt, to say the least, as an uphill run. In my personal opinion, these challenges are at the core of the hurdles many humanities students experience when approaching the digital humanities -- a field deeply interdisciplinary in nature, yet one where technical skills often give an advantage to those coming from STEM backgrounds. And yet, we persevere. I will always remember the hesitation we felt when modelling our first knowledge graph, as well as the gratification of printing our first simple ``Hello world''. While the future can still feel daunting, this journey has given me tools, courage, and a true sense of belonging, enough to embrace what lies ahead with hope and ambition.

\end{spacing}